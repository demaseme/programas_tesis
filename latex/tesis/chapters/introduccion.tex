Un dibujo rectilíneo de una gráfica, también llamado gráfica geométrica, está definido por un conjunto de
puntos en el plano, llamados vértices, y un conjunto de segmentos de recta que unen pares de puntos,
llamados aristas. Un thrackle es un subconjunto de estos segmentos tal que cada par se cruza o tiene un
extremo en común. En el área de geometría combinatoria existe un problema en el que, dada una gráfica, se
busca saber cuál es el mínimo número de thrackles que existen en un conjunto de tal manera que cada
thrackle tiene una de las aristas de un dibujo rectilíneo de la gráfica y no hay dos thrackles del conjunto
que compartan una o más aristas y además, la unión de los thrackles es el conjunto de aristas de la
gráfica. Este parámetro recibe el nombre de anti-thickness geométrico de una gráfica.

El anti-thickness geométrico ha sido estudiado con anterioridad, para gráficas geométricas completas cuyos
vértices están en posición convexa, en el trabajo de \cite{Fabila-Monroy2018}. Denotamos este parámetro
como $At_g(K_n)$; se sabe que \[At_c(K_n) = n - \left\lfloor\sqrt{2n + \frac{1}{4}}-
\frac{1}{2}\right\rfloor.\]

Además, en el trabajo de \cite{Lomeli2018} encuentran el anti-thickness geométrico de una configuración de
puntos conocida como la doble cadena convexa, denotada como $K_{k,l}$. En este trabajo los autores
demuestran que el anti-thickness, $At_g(K_{k,l})$ de la doble cadena convexa es \[ At_g(K_{k,l}) = k+l
-\left\lfloor\sqrt{2l + \frac{1}{4}} -\frac{1}{2}\right\rfloor. \]

En el trabajo de \cite{Dujmovic2017} se ofrecen varios resultados para distintos dibujos de una gráfica, y
además las cotas del anti-thickness geométrico para el caso en el que los puntos de la gráfica completa
están en posición general. Estas cotas son las siguientes:
\[ \frac{n-1}{2} \leq At_g(K_n) \leq n - \left\lfloor\sqrt{2n + \frac{1}{4}}- \frac{1}{2}\right\rfloor.\]

La cota superior proviene de resultado obtenido para el anti-thickness geométrico cuando los puntos están
en posición convexa mientras que la cota inferior proviene del hecho de que un thrackle con el máximo
número de aristas tiene a lo más $n$ aristas. Sin embargo, para dar la cota inferior se asume que cada par
de thrackles con el máximo número de aristas no comparten ni una arista. Esto no es verdad en posición
convexa, lo cual da lugar para cuestionarnos si en realidad esta cota es justa para posición general.

En esta tesis encontramos que en efecto el anti-thickness geométrico $At_g(K_n)$ de una gráfica completa es
$At_g(K_n) = n - \left\lfloor\sqrt{2n + \frac{1}{4}}- \frac{1}{2}\right\rfloor$ para $n\leq 10$. Nosotros
analizamos los dibujos rectilíneos con hasta diez puntos para buscar conjuntos de thrackles usando
algoritmos exhaustivos. También encontramos dibujos rectilíneos cuyos puntos no están en posición convexa
cuyo anti-thickness es igual al proveído en el trabajo de \cite{Fabila-Monroy2018}. Adicionalmente
analizamos la reflexividad, el número de cruce y la convexidad de estos dibujos para trata de
clasificarlos. Encontramos una relación entre estos tres parámetros y el anti-thickness geométrico de la
gráfica completa. Asimismo encontramos el anti-thickness exacto de dibujos rectilíneos en los que no
existen thrackles con $n$ aristas.

Para dar un resultado acerca de un parámetro como el anti-thickness es necesario encontrar propiedades que
sean compartidas por cada uno de los dibujos de una gráfica, podemos encontrar dichas propiedades usando
herramientas combinatorias o geométricas, o bien, analizando los dibujos de una gráfica computacionalmente,
por ello, cada uno de nuestros resultados tiene un algoritmo computacional como sustento. Este enfoque
resulta nuevo para la búsqueda del anti-thickness geométrico ya que, como mencionamos anteriormente, este
problema ha sido atacado usando herramientas geométricas. Nuestros resultados, a pesar de que son para
conjuntos pequeños de puntos, ayudan a entender cómo se comporta el anti-thickness geométrico para puntos
en posición general dando un panorama para conjuntos más grandes de puntos.

La tesis está organizada de la siguiente manera: en el capítulo 1 se explican
a detalle y de manera más formal las definiciones que usamos en este trabajo y que
son necesarias para entender el desarrollo de la tesis, en el capítulo 2 hacemos un recuento de los
resultados obtenidos acerca del anti-thickness geométrico y tratamos de explicar el origen del concepto
tomando en cuenta un problema propuesto con anterioridad, luego, en el capítulo 3, explicaremos los
resultados del trabajo y cómo fueron obtenidos. En dicho capítulo exponemos los algoritmos usados para las
búsquedas exhaustivas. Finalmente, en el capítulo 5 mencionamos nuestras conclusiones y posible trabajo
futuro.
