Considere un dibujo de una gráfica o gráfica geométrica $\mathsf{G}$ como el par de conjuntos $P$, que es un conjunto de $n$ puntos en el plano, y el conjunto $E$ de segmentos de recta que conectan pares de puntos en el plano. Si $E$ contiene los segmentos de recta formados a partir de cada par de puntos de $P$ entonces decimos que $\mathsf{G}$ es completa. Un thrackle es un subconjunto de $E$ junto con los vértices correspondientes de cada segmento, tal que cada par de segmentos se cruza o tiene un extremo en común.

El anti-thickness geométrico es el mínimo número de thrackles que existen, para todos los conjuntos posibles de $n$ puntos, de tal manera que cada segmento de $E$ pertenezca a exactamente un thrackle. Este parámetro se denota como $At_g(K_n)$.

En el trabajo de \cite{Dujmovic2017} se ofrecen varios resultados para distintos dibujos de una gráfica, y
además las cotas del anti-thickness geométrico para el caso en el que los puntos de la gráfica completa
están en posición general. Estas cotas son las siguientes:
\[ \frac{n-1}{2} \leq At_g(K_n) \leq n - \left\lfloor\sqrt{2n + \frac{1}{4}}- \frac{1}{2}\right\rfloor.\]

La cota superior proviene de resultado obtenido para el anti-thickness geométrico cuando los puntos están
en posición convexa mientras que la cota inferior proviene del hecho de que un thrackle con el máximo
número de aristas tiene a lo más $n$ aristas. Sin embargo, para dar la cota inferior se asume que cada par
de thrackles con el máximo número de aristas no comparten ni una arista. Esto no es verdad en posición
convexa, lo cual da lugar para cuestionarnos si en realidad esta cota es justa para posición general.
El anti-thickness geométrico ha sido estudiado con anterioridad, para gráficas geométricas completas cuyos
vértices están en posición convexa, denotamos a esta variación del problema como $At_c(K_n)$. En el trabajo de \cite{Fabila-Monroy2018} prueban que \[At_c(K_n) = n - \left\lfloor\sqrt{2n + \frac{1}{4}}-
\frac{1}{2}\right\rfloor.\]

Además, en el trabajo de \cite{Lomeli2018} encuentran el anti-thickness geométrico de una configuración de
puntos conocida como la doble cadena convexa, este trabajo es interesante ya que trabajan con puntos que no
están en posición convexa. Ellos denotan la doble cadena convexa como $K_{k,l}$. Los autores
demuestran que el anti-thickness, $At_g(K_{k,l})$ de la doble cadena convexa es \[ At_g(K_{k,l}) = k+l
-\left\lfloor\sqrt{2l + \frac{1}{4}} -\frac{1}{2}\right\rfloor. \]

En esta tesis encontramos que el anti-thickness geométrico $At_g(K_n)$ de una gráfica completa es
$At_g(K_n) = n - \left\lfloor\sqrt{2n + \frac{1}{4}}- \frac{1}{2}\right\rfloor$ para $n\leq 10$. Nosotros
analizamos los dibujos rectilíneos con hasta diez puntos para buscar conjuntos de thrackles usando
algoritmos exhaustivos. También encontramos dibujos rectilíneos cuyos puntos no están en posición convexa
y cuyo anti-thickness coincide con el anti-thickness convexo. Adicionalmente
analizamos tres parámetros geométricos de estos dibujos para tratar de clasificarlos. Encontramos una
relación entre dichos parámetros y el anti-thickness geométrico de la gráfica completa. Asimismo
encontramos el anti-thickness exacto de dibujos rectilíneos en los que no existen thrackles con $n$ aristas.

El anti-thickness geométrico ha sido estudiado, como mencionamos anteriormente, para conjuntos en posición
convexa y para la doble cadena convexa. Las propiedades combinatorias de estas configuraciones no dependen
de la posición de los puntos sino del tamaño del conjunto. En este sentido, los resultados actuales del
anti-thickness geométrico están restringidos a estas dos configuraciones de puntos. Hasta el momento, no
hay resultados acerca del anti-thickness geométrico en posición general a parte de las cotas mencionadas
antes. Por ello, resulta interesante estudiar el anti-thickness geométrico para conjuntos de puntos en
posición general no convexa. En nuestro trabajo exploramos este parámetro computacionalmente usando
algoritmos exhaustivos. Este enfoque resulta nuevo para este problema ya que no ha sido estudiado usando
éstas técnicas. Nosotros nos enfocamos en conjuntos con hasta diez puntos ya que los datos que usamos para
este estudio, proveidos en el trabajo de~\cite{Aichholzer2001}, son para conjuntos con este tamaño como
máximo. Además, el número de conjuntos de puntos para cada $n$ crece de manera exponencial, examinarlos de
manera exhaustiva requeriría una amplia cantidad de tiempo de computo. Por ello decidimos restringir la
búsqueda para conjuntos con hasta diez puntos. Sin embargo, estos resultados podrían clarificar los
resultados acerca del anti-thickness geométrico para valores más grandes de $n$.

% Para dar un resultado acerca de un parámetro como el anti-thickness es necesario encontrar propiedades que
% sean compartidas por cada uno de los dibujos de una gráfica, podemos encontrar dichas propiedades usando
% herramientas combinatorias o geométricas, o bien, analizando los dibujos de una gráfica computacionalmente,
% por ello, cada uno de nuestros resultados tiene un algoritmo computacional como sustento. Este enfoque
% resulta nuevo para la búsqueda del anti-thickness geométrico ya que, como mencionamos anteriormente, este
% problema ha sido atacado usando herramientas geométricas. Nuestros resultados, a pesar de que son para
% conjuntos pequeños de puntos, ayudan a entender cómo se comporta el anti-thickness geométrico para puntos
% en posición general dando un panorama para conjuntos más grandes de puntos.
La tesis está organizada de la siguiente manera. En el capítulo 1 se explican
a detalle y de manera más formal las definiciones que usamos en este trabajo y que
son necesarias para entender el desarrollo de la tesis. En el capítulo 2 hacemos un recuento de los
resultados obtenidos acerca del anti-thickness geométrico y tratamos de explicar el origen del concepto
tomando en cuenta un problema propuesto con anterioridad. Luego, en el capítulo 3, explicaremos los
resultados del trabajo y cómo fueron obtenidos, exponemos los algoritmos usados para las
búsquedas exhaustivas. Finalmente, en el capítulo 5 mencionamos nuestras conclusiones y posible trabajo
futuro.
