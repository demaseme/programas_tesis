En esta tesis estudiamos el problema del anti-thickness geométrico de gráficas completas con hasta 10 vértices.

En este trabajo encontramos que el anti-thickness geométrico $At_g(K_n)$ de la gráfica completa de $n$ vértices $K_n$, con $ 3 \leq n \leq 10$, es:
\[ At_g(K_n) =  n - \left\lfloor \sqrt{2n + \frac{1}{4}} - \frac{1}{2}\right\rfloor. \]
Para dar este resultado analizamos las gráficas completas inducidas por los conjuntos de hasta diez
vértices proveídos por la base de datos de tipos de orden del trabajo de~\cite{Aichholzer2001}, encontramos que dos thrackles máximos de la misma gráfica completa comparten al menos una arista, dando así una nueva cota inferior para conjuntos pequeños de puntos. La cota superior está dada por el anti-thickness del dibujo de $K_n$ en posición convexa; nosotros encontramos dibujos, que no están en posición convexa, que tienen el mismo anti-thickness.

Para cada thrackle de las descomposiciones encontradas calculamos el número
de cruce para observar que existe una relación del número de thrackles con número de cruce alto y número de
cruce bajo cercana al 50\%; en este trabajo damos la definición del número de cruce de un thrackle.

También analizamos y listamos cuáles conjuntos de puntos no tienen thrackles máximos y cuáles tienen
solamente un thrackle máximo para cada $n$ en el rango $[3,10]$. Aquí hacemos la observación de que
conjuntos con un número de cruce pequeño tienen menos thrackles máximos que los conjuntos con un número de
cruce elevado con respecto del número de cruce máximo para cada $n$ en el rango anteriormente mencionado.

Para los conjuntos de $n$ puntos, con $3 \leq n \leq 8$ que inducen gráficas completas sin thrackles
máximos analizamos, con un algoritmo exhaustivo, el anti-thickness geométrico de cada uno. Encontramos que
estos dibujos tienen anti-thickness mayor, en una unidad, al anti-thickness geométrico de $K_n$.

Uno de los objetivos de la tesis era obtener el anti-thickness geométrico de $K_n$ con $n\geq 3$. Sin
embargo no fue posible ya que no pudimos probar la generalización del lema~\ref{lema:thdisjuntos} que
establece que la intersección entre dos thrackles máximos de un mismo dibujo siempre sucede. Decidimos no
invertir más tiempo en la generalización del teorema para dar más resultados acerca de conjuntos pequeño de
$n$.

Queremos destacar que el trabajo mostrado en esta tesis muestra el proceso por el cuál pasamos para
intentar resolver el problema del anti-thickness geométrico para conjuntos pequeños de puntos. La cota
inferior del anti-thickness geométrico puede ser obtenida al probar que dos thrackles máximos se
intersectan en al menos una arista o bien, probando que no existen descomposiciones con menos de $n -
\left\lfloor \sqrt{2n + \frac{1}{4}} - \frac{1}{2}\right\rfloor$ thrackles.


\subsubsection{Trabajo futuro}

La generalización del lema~\ref{lema:thdisjuntos} es el trabajo a futuro más fuerte, de encontrarse, el
problema del anti-thickness geométrico para gráficas completas quedaría resuelto de manera general. Esto
deja abierto el problema de encontrar el anti-thickness de un dibujo en específico de manera eficiente.
Pero, como explicamos en el capítulo~\ref{cap3}, encontrar el anti-thickness de una gráfica geométrica equivale a encontrar el número cromático de su gráfica de disyunción y este último es un problema $NP$-difícil~(\cite{Skiena2003}). Sin embargo, existen algoritmos genéticos para encontrar una aproximación al número cromático de una gráfica dada como los presentados en~\cite{Fleurent1996} y \cite{Galinier1999} en los que se muestra que los resultados fueron obtenidos en un rango de tiempo considerablemente bueno para el hardware en el que se implementaron. Es posible implementar estos algoritmos para encontrar la coloración de la gráfica de disyunción y obtener resultados acerca del anti-thickness de algún dibujo de $K_n$.

También es deseable encontrar una manera de caracterizar los dibujos que inducen gráficas completas sin
thrackles máximos, o con solo uno, ya que analizar todas las $\displaystyle \binom{\binom{n}{2}}{n}$
posibles combinaciones de $n$ aristas para buscar thrackles máximos puede resultar ineficiente,
especialmente con conjuntos grandes de puntos.
