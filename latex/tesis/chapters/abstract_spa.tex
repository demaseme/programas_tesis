En el campo de la geometría combinatoria y computacional una gráfica está definida por un conjunto
de vértices $V$ y un conjunto de aristas $E$ de pares de elementos de $V$. Cuando la gráfica tiene
todas las aristas entre pares de vértices decimos que la gráfica es completa. Una gráfica geométrica
es un dibujo de una gráfica en el plano en la que cada vértice es representado por un punto en el
plano y cada arista es representada por un segmento de recta que une dos puntos. Un thrackle
geométrico es una gráfica geométrica en la que cada par de aristas es adyacente o se cruza. El
anti-thickness geométrico de una gráfica es el mínimo número de thrackles que existen en una
descomposición de algún dibujo de la gráfica. Encontrar el anti-thickness de gráficas completas es
un problema abierto y equivale a encontrar el número cromático de una gráfica por lo que es un
problema $NP$-difícil. Actualmente se conoce la cota inferior y la cota superior del anti-thickness
geométrico para gráficas completas. La cota inferior se basa en contar cuántos thrackles son
necesarios en una descomposición de tal manera que la unión de sus aristas sea el conjunto de las
aristas de la gráfica completa y la cota superior se basa en encontrar una descomposición en
thrackles de un dibujo de la gráfica completa cuando sus puntos están en posición convexa. En esta
tesis encontramos el anti-thickness geométrico exacto para gráficas completas con hasta diez
vértices. Usamos algoritmos computacionales para buscar thrackles en dibujos de gráficas completas
y observamos que cada par de thrackles máximos se intersecta en al menos una arista lo que nos
permitió probar que la cota inferior no es justa para gráficas con hasta diez vértices. Además
hicimos observaciones con respecto a otras propiedades geométricas de los thrackles y las
gráficas completas como el número de cruce y la reflexividad de un conjunto de puntos.
