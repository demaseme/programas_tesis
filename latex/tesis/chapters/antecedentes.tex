En este capítulo daremos las definiciones necesarias para el desarrollo del proyecto.
Empezamos estableciendo conceptos relacionados con gráficas abstractas y después
hablaremos de gráficas en el plano, explicamos el concepto de tipo de orden y cómo
se utiliza en este trabajo. Finalmente hablaremos del anti-thickness abstracto
y del anti-thickness geométrico.

La estructura más básica que se utiliza en nuestro trabajo son las gráficas abstractas
a las que llamaremos solamente gráficas. La definición se sigue:
\section{Gráfica}
Una gráfica $G$ está compuesta por un conjunto $V$ no vacío de objetos a los que llamamos vértices
y un conjunto $E$ de parejas de elementos de $V$ a los que llamamos aristas. Denotamos
a la arista $e$ compuesta por los vértices $u$ y $v$ como $(u,v)$. Para describir a la gráfica $G$
compuesta por el conjunto $V$ de vértices y el conjunto $E$ de aristas escribimos $G=(V,E)$.
Para referirnos al conjunto de vértices de $G$ escribimos $V(G)$ y para referirnos
al conjunto de aristas de $G$ escribimos $E(G)$.
\section{Gráfica geométrica}
\section{Thrackles}
\section{Tipo de Orden}
\section{Anti-thickness y anti-thickness geométrico}
