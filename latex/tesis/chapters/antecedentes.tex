En este capítulo daremos las definiciones necesarias para el desarrollo del proyecto.
Empezamos estableciendo conceptos relacionados con gráficas abstractas y después
hablaremos de gráficas en el plano, explicamos el concepto de tipo de orden y cómo
se utiliza en este trabajo. Finalmente hablaremos del anti-thickness abstracto
y del anti-thickness geométrico.

La estructura más básica que se utiliza en nuestro trabajo son las gráficas abstractas
a las que llamaremos solamente gráficas. La definición se sigue:
\section{Gráfica}
La siguiente definición fue tomada de~\cite{Chartrand2008}.

Una gráfica $G$ está compuesta por un conjunto $V$ no vacío de objetos a los que llamamos vértices
y un conjunto $E$ de parejas de elementos de $V$ a los que llamamos aristas. Denotamos
a la arista $e$ compuesta por los vértices $u$ y $v$ como $(u,v)$. Para describir a la gráfica $G$
compuesta por el conjunto $V$ de vértices y el conjunto $E$ de aristas escribimos $G=(V,E)$.
Para referirnos al conjunto de vértices de $G$ escribimos $V(G)$ y para referirnos
al conjunto de aristas de $G$ escribimos $E(G)$.

Decimos que dos vértices $u,v\in V(G)$ son \emph{adyacentes} si existe una arista
$(u,v)\in E(G)$. Una gráfica es \emph{completa} si cada pareja de vértices
en la gráfica es adyacente. Para referirnos a la gráfica completa de $n$ vértices
escribimos $K_n$. Decimos que dos aristas $e_1,e_2 \in E(G)$ son adyacentes
si inciden en el mismo vértice. Una gráfica $G$ es \emph{bipartita} si es posible
dar una partición de $V(G)$ en dos subcojuntos $U$ y $W$ de tal manera que cada
arista de $G$ tiene un extremo en $U$ y otro extremo en $W$. 
\section{Gráfica geométrica}
La siguiente definición fue tomada de~\cite{Pach2013}.
Un \emph{dibujo} de una gráfica es una representación de la gráfica en el plano de
tal manera que los vértices son representados por puntos en posición general
y las aristas como curvas simples continuas que conectan pares de vértices.

Cuando restringimos dichas curvas a segmentos de recta el dibujo de la gráfica
adquiere el nombre de \emph{gráfica geométrica}. Sin perdida de generalidad
representamos a una gráfica geométrica como $\mathsf{G}$ y nos referimos a su
conjunto de vértices como $V(\mathsf{G})$ y a su conjunto de aristas como
$E(\mathsf{G})$.

\section{Thrackles}
\section{Tipo de Orden}
\section{Anti-thickness y anti-thickness geométrico}
