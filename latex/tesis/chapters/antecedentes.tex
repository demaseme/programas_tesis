En este capítulo damos algunas definiciones necesarias para presentar lo que
se conoce como descomposición de gráficas completas en thrackles.
Empezamos estableciendo conceptos relacionados con gráficas abstractas y después
hablaremos de gráficas en el plano, posteriormente explicamos el concepto de tipo de orden y cómo
se utiliza en este trabajo, continuamos hablando del anti-thickness abstracto
y del anti-thickness geométrico y finalmente explicamos el número cromático de una gráfica.

\section{Gráficas}
El concepto base del cual se desprenden otras definiciones es el de gráfica.
Todas las definiciones que presentamos en esta sección fueron tomadas de~\cite{Chartrand2008}.

Una \emph{gráfica} $G$ está compuesta por un conjunto no vacío $V$ de objetos a los que llamamos \emph{vértices}
y por un conjunto $E$, de parejas de elementos de $V$, a los que llamamos \emph{aristas}. Denotamos
a la arista $e$ compuesta por los vértices $u$ y $v$ como $(u,v)$. Para describir a la gráfica $G$
compuesta por el conjunto $V$ de vértices y el conjunto $E$ de aristas escribimos $G=(V,E)$.
Para referirnos al conjunto de vértices de $G$ escribimos $V(G)$ y para referirnos
al conjunto de aristas de $G$ escribimos $E(G)$. En la figura~\ref{fig:g5vex} presentamos
un ejemplo de una gráfica.

\begin{figure}[t]
  \centering
  \includegraphics[width=0.27\linewidth]{g5vex.png}
  \caption{Una gráfica de cinco vértices y cuatro aristas. Los vértices $v_1$ y $v_2$
  son adyacentes y las aristas $(v_1,v_3)$ y $(v_2,v_5)$ son adyacentes.}
  \label{fig:g5vex}
\end{figure}

Decimos que \emph{dos vértices $u,v\in V(G)$ son adyacentes} si existe la arista
$(u,v)\in E(G)$. La figura~\ref{fig:g5vex} muestra un ejemplo de dos vértices adyacentes.
% \begin{figure}[htb]
%   \centering
%   \includegraphics[width=0.3\linewidth]{exady}
%   \caption{En esta gráfica el vértice $1$ es adyacente con el vértice $2$ pero
%   no es adyacente con el vértice $3$.}
%   \label{fig:exady}
% \end{figure}
Decimos que \emph{dos aristas $e_1,e_2 \in E(G)$ son adyacentes}
si inciden en el mismo vértice. Una gráfica es \emph{completa} si cada pareja de vértices
en la gráfica es adyacente. Mostramos un ejemplo de adyacencia de
aristas en la figura~\ref{fig:g5vex} y un ejemplo de una gráfica completa en la figura~\ref{fig:excomplete}.
Para referirnos a la gráfica completa con $n$
vértices escribimos $K_n$. Una gráfica $G$ es \emph{bipartita} si es posible
dar una partición\footnote{Una partición $P$ de un conjunto $X$ es
una colección de subconjuntos que cumplen lo siguiente:
\begin{itemize}
\item Ningún elemento de $P$ es el conjunto vacío.
\item La unión de todos los elementos de $P$ es exactamente el conjunto $X$.
\item La intersección de cualesquiera dos elementos de $P$ es vacía.
\end{itemize} }
de $V(G)$ en dos subcojuntos $U$ y $W$ de tal manera que cada
arista de $G$ tenga un extremo en $U$ y otro extremo en $W$. Presentamos
un ejemplo de gráfica bipartita en la figura~\ref{fig:exbipar}.
\begin{figure}[htb]
  \centering
  \includegraphics[width=.4\linewidth]{excomplete}
  \caption{La gráfica completa con 6 vértices tiene una arista por cada par de vértices.}
  \label{fig:excomplete}
\end{figure}
\begin{figure}[h]
  \centering
  \includegraphics[width=0.3\linewidth]{exbipartita}
  \caption{Un ejemplo de una gráfica bipartita con partición $U$ y $W$, ambos conjuntos son de tamaño tres.}
  \label{fig:exbipar}
\end{figure}

Una \emph{descomposición} $D$ de una gráfica $G$ es una colección $D=\{G_1,G_2,\dots,G_k\}$ de
subgráficas de $G$ que cumple con dos condiciones:
\begin{enumerate}
  \item Ninguna subgráfica $G_i$ contiene vértices aislados.
  \item Cada arista de $G$ pertenece a exactamente una subráfica $G_i$ de $D$.
\end{enumerate}

La figura~\ref{fig:exdecom} ilustra un ejemplo de una descomposición de la gráfica $K_4$.
\begin{figure}[htbp]
  \centering
  \includegraphics[width=0.8\linewidth]{exdecom}
  \caption{Un ejemplo de una descomposición $D$ de $K_4$ en dos gráficas.
  Aquí $D=\{G_1,G_2\}$ donde $G_1$ es la gráfica compuesta por las aristas $(1,4),(1,2),(2,3),(3,4)$ y
  $G_2$ es la gráfica compuesta por las aristas $(1,3),(2,4)$.}
  \label{fig:exdecom}
\end{figure}
%
% Introducir la próxima sección
%

En este trabajo hacemos descomposiciones de gráficas (abstractas) en gráficas
geométricas con cierta propiedad que explicamos más adelante, en la siguiente
sección exponemos el concepto de dibujo de una gráfica y gráfica geométrica.
\subsection{Gráfica geométrica}
En esta sección abordamos uno de los conceptos clave del trabajo que son
las gráficas geométricas, empezamos explicando el significado de un
\emph{dibujo} de una gráfica (abstracta) para continuar con la descripción
de un dibujo de una gráfica con características especiales al que llamamos
\emph{gráfica geométrica}.

Los primeros dos párrafos de esta sección fueron tomados de~\cite{Pach2013}. El tercer párrafo
fue extraido de~\cite{Lara2019}. El cuarto párrafo fue tomado de~\cite{Pach2011}.
Un \emph{dibujo} $\mathsf{G}=(\mathsf{V},\mathsf{E})$ de una gráfica $G$ es una representación de la
gráfica $G$ en el plano tal que 1) cada vértice de $G$ es representado por un punto en el plano
y 2) cada arista de $G$ es representada como una curva simple continua que conecta un par de puntos.
El conjunto de vértices $V$ y el conjunto de aristas $E$ de $\mathsf{G}$
son los puntos y las curvas, respectivamente. Sin perdida de generalidad nos referimos al
conjunto de puntos de $\mathsf{G}$ como $V(\mathsf{G})$, y les llamamos vértices, y nos
referimos al conjunto de curvas de $\mathsf{G}$ como $E(\mathsf{G})$, y les llamamos aristas.

Cuando restringimos las curvas que representan a las aristas del dibujo de $G$
a segmentos de recta llamamos al dibujo de la gráfica \emph{gráfica geométrica}.
Una gráfica geométrica es completa si existe un segmento de recta entre cada par de vértices
de $V(\mathsf{G})$. En la figura~\ref{fig:exdrawk5} mostramos un dibujo de $K_5$ y una
gráfica geométrica de $K_5$.
\begin{figure}[htpb]
  \centering
  \includegraphics[width=0.8\linewidth]{exdrawk5}
  \caption{A la izquierda observamos un dibujo de $K_5$ y a la derecha observamos
  una gráfica geométrica de $K_5$.}
  \label{fig:exdrawk5}
\end{figure}

Sea $S$ un conjunto de $n$ puntos en posición general en el plano y sea $\mathsf{G}$
una gráfica geométrica de $G$. Decimos que $\mathsf{G}$ está definida sobre $S$ si $V(\mathsf{G}) = S$.
Cualquier conjunto $S$ de puntos en posición general induce una gráfica geométrica completa.

Decimos que dos aristas $e_1,e_2 \in E(\mathsf{G})$ se \emph{cruzan} si existe un punto $p$,
en alguna de las aristas, tal que en $p$ la arista $e_1$ pasa de un lado de la arista
$e_2$ hacia el otro lado. Decimos que dos aristas $e_1, e_2 \in E(\mathsf{G})$ son
\emph{adyacentes} si comparten un vértice.
En este trabajo decimos que dos aristas de una gráfica geométrica se \emph{intersectan}
si son adyacentes o si se cruzan.
Mostramos un ejemplo de intersección de aristas en la figura~\ref{fig:exintersection}.
\begin{figure}[htpb]
  \centering
  \includegraphics[width=0.3\linewidth]{exintersection}
  \caption{En este ejemplo la arista $(1,2)$ no se intersecta con la arista $(3,4)$
  (son disjuntas) pero sí se intersecta con la arista $(2,5)$.
  La arista $(2,5)$ se cruza con la arista $(3,4)$ y por lo tanto se intersectan.}
  \label{fig:exintersection}
\end{figure}
% Sin perdida de generalidad
% representamos a una gráfica geométrica como $\mathsf{G}$ y nos referimos a su
% conjunto de vértices como $V(\mathsf{G})$ y a su conjunto de aristas como
% $E(\mathsf{G})$.

El concepto que estudiamos en esta tesis está ligado a gráficas geométricas
donde cada par de aristas se intersectan una vez, estas gráficas geométricas
reciben el nombre de thrackles. En la siguiente sección explicamos qué son
los thrackles formalmente.

\subsection{Thrackles}
Sea $\mathsf{G}$ un dibujo de una gráfica $G$. Decimos que $\mathsf{G}$ es un
\emph{thrackle} si cada par de aristas se intersecta exactamente una vez.
Los thrackles fueron definidos por John Conway en la decada de 1960~(\cite{Pach2013}).
Conway también conjeturó que el número de aristas en un thrackle no puede
exceder el número de sus vértices~(\cite{Fulek2011}).
Un thrackle de $n$ vértices es \emph{máximo} si tiene exactamente
$n$ aristas. La figura~\ref{fig:exmaxth} muestra un thrackle máximo.

\begin{figure}[htpb]
  \centering
  \includegraphics[width=0.35\linewidth]{exmaxth}
  \caption{Un thrackle máximo sobre un conjunto de seis vértices.}
  \label{fig:exmaxth}
\end{figure}

Un thrackle en el que todas sus aristas son segmentos de recta es conocido como
\emph{thrackle geométrico}(\cite{Schaefer2018}). En la figura~\ref{fig:exthgeotop}
presentamos un ejemplo de thrackle y un ejemplo de thrackle geométrico.
En este trabajo nos referimos a los thrackles geométricos
como thrackles ya que solo estudiamos descomposiciones de gráficas con
thrackles geométricos.

\begin{figure}[htb]
  \centering
\begin{subfigure}[h]{.4\textwidth}
  \centering
  \includegraphics[width=.6\linewidth]{exthtop}
  \caption{Un thrackle con cinco vértices.}
  \label{fig:exthtop}
\end{subfigure}\hfill%
\begin{subfigure}[h]{.4\textwidth}
  \centering
  \includegraphics[width=.6\linewidth]{exthgeo}
  \caption{Un thrackle geométrico con cinco vértices.}
  \label{fig:exthgeo}
\end{subfigure}
\caption{Ambas figuras ilustran thracklese definidos sobre el mismo conjunto de
puntos. En los dos casos el thrackle dibujado es máximo.}
\label{fig:exthgeotop}
\end{figure}

Una gráfica (abstracta) $G$ es \emph{thrackleable} si puede ser dibujada en el plano como un thrackle.

Una descomposición por thrackles $D$, de una gráfica geométrica
$\mathsf{G}$, es una colección $D=\{\mathsf{G}_1,\mathsf{G}_2,\dots,\mathsf{G}_k\}$
de subgráficas geométricas que cumple con tres condiciones:
\begin{enumerate}
  \item Cada subgráfica $\mathsf{G}_i$ es un thrackle.
  \item Ninguna subgráfica $\mathsf{G}_i$ contiene vértices aislados.
  \item Cada arista de $\mathsf{G}$ pertenece a exactamente una subráfica $\mathsf{G}_i$ de $D$.
\end{enumerate}

Dada una gráfica completa (abstracta) esta puede ser dibujada en el plano
de muchas maneras, solo basta con mover un punto en cualquier dirección
para obtener diferentes dibujos de la misma gráfica completa, de hecho
hay un número infinito de dibujos para una sola gráfica abstracta. Estudiarlos
todos no es posible y por ello necesitamos discretizar el número de
posibles dibujos geométricos para una sola gráfica. El tipo de orden es
la herramienta que otorga un número finito de dibujos combinatoriamente diferentes para
gráficas abstractas y lo explicamos a continuación.

\subsection{Tipo de Orden}

Para entender cómo funciona el tipo de orden de un conjunto de puntos
debemos definir la orientación de una tripleta de puntos.

Tres puntos en el plano en posición general pueden tener una orientación en sentido horario
o una orientación en sentido anti-horario. Si la tripleta está orientada en sentido horario
asignamos a esa tripleta un valor negativo. Si la tripleta está orientada en sentido anti-horario
asignamos a esa tripleta un valor positivo. En la figura~\ref{fig:triplet} mostramos un ejemplo de
cada orientación posible para una tripleta.
\begin{figure}[htpb]
  \centering
  \includegraphics[width=0.36\linewidth]{triplet}
  \caption{Esta figura muestra las posibles orientaciones de una tripleta de puntos. La tripleta $\{p,q,w\}$ tiene
  asignado el valor de $(-)$ porque $w$ está a la derecha del segmento formado por los puntos $p,q$. La tripleta $\{p,q,u\}$ tiene
  asignado el valor de $(+)$  porque $u$ está a la izquierda del mismo segmento.}
  \label{fig:triplet}
\end{figure}

Las definiciones siguientes fueron tomadas de~\cite{Aichholzer2002}.
El tipo de orden de un conjunto $S$ de $n$ puntos en posición general, digamos
$S=\{p_1,p_2,\dots,p_n\}$ es una función que asigna a cada tripleta ordenada
$i,j,k \in \{1,2,\dots,n\}$ la orientación de la tripleta de puntos $\{p_i,p_j,p_k\}$.

Decimos que dos conjuntos $S_1$ y $S_2$ son \emph{combinatoriamente equivalentes} si
tienen el mismo tipo de orden de otra forma, si no son equivalentes decimos que son
\emph{combinatoriamente distintos}. Si $S_1$ y $S_2$ son combinatoriamente equivalentes
dos segmentos en $S_1$ se cruzan si y solo si los segmentos correspondientes en $S_2$
se cruzan. Solo hay una manera (combinatoriamente equivalente) de acomodar tres puntos,
dos maneras de acomodar cuatro puntos y tres maneras de acomodar cinco puntos.
En la figura~\ref{fig:exotk5} presentamos conjuntos
combinatoriamente equivalentes de 5 puntos. En la figura~\ref{fig:ot5}
mostramos los diferentes tipos de orden para un conjunto de 5 puntos.

\begin{figure}[htpb]
  \centering
  \includegraphics[width=0.8\linewidth]{exotk5}
  \caption{Estos dos conjuntos de puntos tienen el mismo tipo de orden.
  Observe que el valor de cada una de las tripletas del dibujo
  que está a la izquierda es igual al valor de las tripletas del
  dibujo a la derecha.}
  \label{fig:exotk5}
\end{figure}
\begin{figure}[htpb]
  \centering
  \includegraphics[width=0.8\linewidth]{ot5}
  \caption{Las tres maneras diferentes de distribuir 5 puntos en el plano. Cualquier
  otra configuración es equivalente a alguna de estas tres configuraciones.}
  \label{fig:ot5}
\end{figure}

No es trivial enumerar o contar los conjuntos combinatoriamente diferentes, por ejemplo,
dados $n$ puntos podemos colocarlos en posición convexa y obtener el primer tipo de orden
para $n$ puntos, luego podemos colocar $n-1$ puntos en posición convexa y un punto dentro
del $n-1$-gono y evaluar de cuántas maneras combinatoriamente distintas es posible colocar
un punto dentro del polígono. Posteriormente podemos ver que pasa con $n-2$ puntos
en posición convexa y dos dentro y así sucesivamente hasta que tengamos 3 puntos en posición
convexa y $n-(n-3)$ dentro del triángulo. Usando una técnica parecida se sabe que para $n=10$
hay más de 10 millones de tipos de orden.

En el trabajo de~\cite{Aichholzer2002} ofrecen una base de datos que contiene los conjuntos
combinatoriamente diferentes para $n\leq 10$. En la tabla~\ref{tab:ots} se presenta el número
de conjuntos diferentes para cada $n$ y el tamaño en bytes de la base de datos.


Es imposible analizar todos los posibles conjuntos de $n$ puntos en el plano
ya que existe un número infinito de ellos. Sin embargo, para $n\leq 10$, el trabajo de~\cite{Aichholzer2002}
nos otorga un número finito de conjuntos de $n$ puntos combinatoriamente diferentes.
Analizar los tipos de orden proveidos es equivalente a analizar todos los conjuntos de $n$ puntos
en el plano para $n\leq 10$.

En este trabajo buscamos descomposiciones de gráficas geométricas completas en thrackles.
Y como se mencionó antes, un conjunto de $n$ puntos en posición general
induce una gráfica completa de $n$ vértices en el plano. Nosotros analizamos
cada tipo de orden para cada $n\leq 10$: inducimos la gráfica completa de $n$ vértices,
examinamos sus thrackles y luego buscamos una descomposición. Cuando buscamos
una descomposición por thrackles de una gráfica minimizando el número de thrackles utilizados
estamos buscando el anti-thickness de la gráfica. Dicho concepto será explicado
formalmente en seguida.
\begin{table}[ht]
  \centering
  \begin{tabular}{|c|c|r|}
  \hline
  $n$ & Número de conjuntos & Tamaño    \\ \hline
  3     & 1                   & 6       \\ \hline
  4     & 2                   & 16      \\ \hline
  5     & 3                   & 30      \\ \hline
  6     & 16                  & 192     \\ \hline
  7     & 135                 & 1890    \\ \hline
  8     & 3315                & 53040   \\ \hline
  9     & 158817              &	5 717 412   \\\hline
  10    & 14309547            & 572 381 880 \\ \hline
  \end{tabular}
  \caption{Tipos de orden para cada $n\leq10$.}
  \label{tab:ots}
\end{table}

\section{Anti-thickness y anti-thickness geométrico}
Las siguientes definiciones fueron tomadas de~\cite{Dujmovic2017}.
\begin{definition}{\emph{Anti-thickness de una gráfica.}}
  El anti-thickness de una gráfica $G$ es el entero $k$ más pequeño tal que existe una
  partición de $E(G)$, de tamaño $k$, en la que cada elemento de la partición
  es una gráfica thrackleable.
\end{definition}
La figura~\ref{fig:exantithickness} ilustra un ejemplo del anti-thickness de $K_5$.
\begin{figure}[htpb]
  \centering
  \includegraphics[width=0.7\linewidth]{exantithickness}
  \caption{La figura muestra a $K_5$ a la izquierda y a la derecha dos thrackles
  cuya unión es $K_5$. Consideremos las aristas de la gráfica completa inducida por los vértices $1,2,3,4$,
  estas aristas inducen un thrackle mientras que las aristas con un extremo en el vértice $5$ inducen otro thrackle.
  El anti-thickness de $K_5$ es precisamente igual a dos. Este resultado se discute en el capítulo de resultados.}
  \label{fig:exantithickness}
\end{figure}

Cuando deseamos que los thrackles usados en la descomposición de la gráfica sean
específicamente geométricos, debemos buscar el anti-thickness geométrico. Esta
es la principal diferencia entre el anti-thickness y el anti-thickness geométrico.
Adicionalmente en el anti-thickness los vértices no requieren ser congruentes
de acuerdo a las posiciones en el plano de los thrackles de la descomposición
mientras que en el anti-thickness geométrico si se requiere la congruencia de la
posición de los vértices entre las thrackles de la descomposición. A continuación
definimos el anti-thickness geométrico de una gráfica y posteriormente presentamos
un ejemplo.

\begin{definition}{\emph{Anti-thickness geométrico de una gráfica.}}
El anti-thickness geométrico de una gráfica $G$ es el entero $k$ más pequeño tal que
existe un dibujo $\mathsf{G}$ de $G$ para el cual hay una partición de $E(\mathsf{G})$,
de tamaño $k$, en la que cada elemento de la partición induce un thrackle.
\end{definition}
En la figura~\ref{fig:exantithicknessgeo} damos un ejemplo del anti-thickness geométrico de $K_5$.
\begin{figure}[htpb]
  \centering
  \includegraphics[width=0.6\linewidth]{exantithicknessgeo}
  \caption{En esta figura podemos observar una descomposición de $K_5$ en 3 thrackles
  geométricos. Esta descomposición se muestra en la figura del lado derecho, cada
  thrackle está dibujado con diferentes patrones de línea. El anti-thickness geométrico
  de $K_5$ es exactamente tres, esto es demostrado en la sección de resultados.} %FALTA CITAR <<<<<<<<<<<<<<<<<<<<<<<
  \label{fig:exantithicknessgeo}
\end{figure}

Dar una descomposición de una gráfica en thrackles es equivalente a encontrar
conjuntos independientes de aristas que comparten ciertas propiedades. A su vez,
encontrar los conjuntos independientes de una gráfica está ligado a encontrar el \emph{número cromático}
de una gráfica (abstracta). Por ello es pertinente estudiar este concepto.

\section{Número cromático}
Las siguientes definiciones fueron tomadas de~\cite{Chartrand2008}.

Una \emph{coloración propia} de los vértices de una gráfica $G$ es la asignación
de colores a los vértices de $G$ tal que cada vértice tiene un solo color
asignado y dos vértices adyacentes tienen diferentes colores.
Un color puede ser un color como rojo, verde, amarillo, etc. cuando el número
de colores a usar es pequeño, de otra forma se usan enteros $1,2,\dots,k$
para algún entero positivo $k$. Si la coloración propia usa $k$ colores
diferentes de esta manera decimos que tenemos una \emph{k-coloración} de la gráfica $G$.
Dada una $k$-coloración de una gráfica $G$, sea $V_i$ el conjunto de vértices
de $G$ que tienen el color $i$ asignado llamamos a $V_i$
una \emph{clase cromática}. El conjunto $\{V_1,V_2,\dots,V_k\}$ genera una \emph{partición}
en conjuntos independientes de los vértices de $G$.

Una gráfica $G$ es \emph{k-colorable} si existe una coloración propia de $G$ de tamaño $k$.
El entero positivo $k$ más pequeño para el cual $G$ es $k$-colorable recibe el nombre
de \emph{número cromático} de $G$. Lo denotamos como $\chi(G)$.
% Si consideramos cualquier partición de $V(G)$, el número cromático $\chi(G)$ de $G$ es la
% cardinalidad más pequeña posible de dicha partición.
% El número cromático
% de $G$ es el mínimo número de conjuntos independientes que pueden existir en una partición
% de $V(G)$.

La figura~\ref{fig:excoloring} muestra un ejemplo de una coloración propia de una gráfica $G$.
En este ejemplo ilustramos cada clase cromática dibujando los vértices con diferentes colores
representados por una cruz, un círculo y un cuadrado.
\begin{figure}[htpb]
  \centering
  \includegraphics[width=0.25\linewidth]{excoloring}
  \caption{Una coloración propia de una gráfica $G$. Esta coloración es de tamaño 3, por
  lo tanto decimos que es una 3-coloración de $G$. Para esta gráfica en particular
  no existe una coloración más pequeña, por lo que su número cromático es 3. Nótese
  que los conjuntos independientes son formados por vértices que tienen el mismo
  color asignado y no son adyacentes entre sí.}
  \label{fig:excoloring}
\end{figure}
