Combinatorial geometry is the area of discrete math and computing which studies combinatoric properties of geometrical objects with discrete features (such as points, planes, lines or geometric graphs). It's envolved in coloring, symmetry, partition and decomposition problems of said geometrical objects. In this thesis we work with a combinatorial geometry problem, particularly with geometric graphs from which we wish to study a parameter currently known as geometric anti-thickness. We say a drawing or a geometric graph is a set of points in the plane and a set of straight lines connecting pairs of points, called edges. A thrackle is a subset of said edges and its corresponding extreme points such that each pair of edges in the subset either crosses or share an extreme point, a thrackle is maximum when it has the maximum number of edges. Given a graph, if we found the minimal number of thrackles, for every drawing of the graph, such that every edge belongs to exactly one thrackle, we would've found the geometric anti-thickness of the given graph.
Finding the geometric anti-thickness is an open problem in combinatorial geometry. Currently, the lower and upper bounds are known. The lower bound is based on counting how many thrackles in a set are necessary such that the previous conditions hold. The upper bound is based on finding a set of thrackles with the previously mentioned conditions of a drawing of the complete graph when its points are in convex position. Note that in the lower bound it is assumed that every pair of maximal thrackles do not share an edge, this is not true for convex position. In the case of the upper bound, the result was achieved by using only one drawing (convex) and this does not give insight about geometric anti-thickness when the drawing is in non-convex general position. In this thesis we find the exact geometric anti-thickness for complete graphs with up to ten vertices We use algorithms to find thrackles in complete graph drawings, we observed that every pair of maximal thrackles share at least one edge. This allowed us to proove that the lower bound it's not tight for graphs with up to ten vertices. Besides, we made observations respecting other geometric properties of thrackles and complete graphs to find a relation between these and geometric anti-thickness.
