Antes de hablar del anti-thickness y los resultados, es sensato hablar de otro
concepto llamado \textbf{thickness} pues su definición está estrechamente
relacionada con el anti-thickness. En el artículo de Eppstein y otros
\cite{Dillencourt2004} se recupera la definición de thickness teórico y thickness
geométrico, además proporciona cotas para éste último. Podemos definir el
thickness de una gráfica en términos del índice crómático de una gráfica y
mantenerlo coherente con la definición de Eppstein: $\theta(G):$ Mínimo $k$
tal que existe una partición de las aristas de $G$, de tamaño $k$ en gráficas planares.

Por otro lado el thickness geométrico $\bar{\theta}(G)$: Mínimo $k$ tal que
existe un dibujo geométrico $\mathsf{G}$ de $G$ cuyas aristas pueden ser
particionadas en $k$ gráficas planas. Eppstein prueba que:
\[ \bar{\theta}(G) \in [\Bigl\lceil n/5.646 + 0.342 \Bigr\rceil, \Bigl\lceil\frac{n}{4}\Bigr\rceil] \]
y encuentra el valor exacto para $K_n$ con $n\leq 12$ y $n\in\{15,16\}$. También
estudia el anti-thickness geométrico para gráficas completas bipartitas y demuestra:
\[
  \Bigl\lceil \frac{ab}{2a+2b-4} \Bigr\rceil \leq \theta(K_{a,b}) \leq \bar{\theta}(K_{a,b})
  \leq \Bigl\lceil \frac{min(a,b)}{2} \Bigr\rceil
\]
Es natural pensar que si una gráfica puede ser descompuesta en sub-gráficas donde
no existe ningún cruce entre dos aristas, es quizás posible descomponer la
gráfica en sub-gráficas donde siempre ocurran cruces entre dos aristas, es decir
en thrackles; este concepto es conocido como el anti-thickness de una gráfica y
analogo al caso del thickness también existe el anti-thickness geométrico.

Antes de examinar los resultados del anti-thickness, resulta interesante verificar
el trabajo de Araujo y Urrutia \cite{Araujo2005}; aquí se define la gráfica de
disyunción $D(S)$ de un conjunto $S$ de puntos en el plano, donde considerando
todos los segmentos de recta entre pares de vértices de $S$ como los vértices
de $D(S)$, existe una arista entre dos vértices si los dos segmentos
de recta correspondientes se cruzan o comparten un vértice.

Si asignamos un color a cada vértice de $D(S)$ de tal manera que dos
vértices adyacentes no compartan color y además minimizamos el número de colores
usados, obtenemos el número cromático de $\chi(D(S))$. Es fácil ver que una
clase cromática de $D(S)$ es un conjunto independiente y además un thrackle. Si
ahora consideramos todos los posibles encajes de $S$ en el plano y para cada uno
encontramos $\chi(D(S))$, tendremos una lista de enteros que representa el tamaño
mínimo de la descomposición en thrackles de una gráfica completa cuyos
vértices están dados por $S$ en el plano. Si de esta lista consideramos ahora
el valor mínimo encontraremos el anti-thickness para la gráfica completa geométrica de
$n$ vértices. Sin embargo, en este trabajo se considera el máximo valor de dicha lista;
definiendo entonces :
\[ d(n) = \max\{\chi(D(S)): S \subset \mathbf{R}^2 \text{ en posición general }, |S|=n\}\]

De manera concreta, el trabajo presenta cotas para $d(n)$, y $d_c(n)$ que se define
de manera similar pero con $S$ en posición convexa.

La diferencia entre $d(n)$ y el anti-thickness reside en que el último buscamos el
mínimo de la lista de números cromáticos de $D(S)$, para ver más claramente
la relación entre $d(n)$ y el anti-thickness podemos definir el
anti-thickness geométrico de la gráfica completa de n vértices $at(n)$ como sigue :
\[ at(n) = \min\{\chi(D(S)): S \subset \mathbf{R}^2 \text{ en posición general }, |S|=n\}\]

Por otro lado, debido a que cuando $K_n$ es encajado en el plano en posición convexa
los dibujos posibles no presentan diferencias combinatorias, se tiene que el
anti-thickness geométrico convexo de la gráfica completa de $n$ vértices ($cat(n)$) :
\[ d_c(n) = cat(n)\]

Como se mencionó anteriormente el anti-thickness está relacionado con la descomposición
de una gráfica en sub-gráficas donde siempre ocurren cruces, podemos ver el
anti-thickness como el lado opuesto del thickness. Dujmovic y Wood \cite{Dujmovic2017} presentan resultados
del anti-thickness para familias específicas de gráficas, por ejemplo árboles, 2-tracks, books, gráficas
outerplanar, k-queues, entre otros haciendo observaciones de resultados anteriores.
Además se prueba la relación entre thickness y anti-thickness, concretamente prueban que
para toda gráfica con anti-thickness $k$ y thickness $t$
: \[ k \leq t \leq \Bigl\lceil \frac{3k}{2}\Bigr\rceil \]

Acerca del anti-thickness (teórico) de gráficas completas, en~\cite{Dujmovic2017}
se prueba que es al menos $\frac{n}{3}$ y a lo más $\lceil \frac{n-1}{2} \rceil$.
Sin embargo ellos conjeturan que el anti-thickness de una gráfica completa es
exactamente $\lceil \frac{n-1}{2} \rceil$.
% Los resultados presentados en ese trabajo que tienen que ver con el anti-thickness
% geométrico de una gráfica completa se enfocan  en el caso específico de puntos en posición convexa.


El principal resultado para posición convexa lo presenta Wood y Fabila~\cite{Fabila-Monroy2018}.
Aquí se prueba que cuando los puntos están en posición convexa dos thrackles máximos
comparten al menos una arista y por lo tanto la unión de $k$ thrackles máximos tiene
a lo más $kn - \binom{k}{2}$ aristas. Luego, como una gráfica completa de $n$
vértices tiene $\binom{n}{2}$ aristas la resolución de la desigualdad
\[ \binom{n}{2} \leq kn - \binom{k}{2} \]
otorga el resultado de la cota inferior para el anti-thickness geométrico convexo que es
\[ n - \Bigl\lfloor \sqrt{2n + \frac{1}{4}} - \frac{1}{2} \Bigr\rfloor \]

La cota superior se obtiene dando una coloración propia de los vértices de la gráfica
de disyunción $D(S)$. La coloración es lograda consiguiendo trazar caminos en una
estructura conocida como polyomino en la que los vértices de $D(S)$ son las filas
y las columnas de dicha estructura. En este trabajo cada camino representa un
conjunto independiente de vértices en $D(S)$ y por lo tanto un thrackle. Concluyen
dando el número máximo de caminos posibles en el polyomino, dicho número coincide con
el de la cota inferior y luego se tiene que el anti-thickness geométrico convexo ($cat(n)$) es:
\[ n - \Bigl\lfloor \sqrt{2n + \frac{1}{4}} - \frac{1}{2} \Bigr\rfloor \]

Es importante notar que el anti-thickness geométrico convexo es acotado examinando el número mínimo
y máximo de aristas aportadas por una colección de thrackles máximos, dicho de
otra manera se basa en descomponer la gráfica completa convexa en thrackles máximos.
Esta idea fue retomada en este trabajo para intentar ajustar las cotas del anti-thickness
en posición general.

Con respecto a puntos que no están estrictamente en posición convexa Leaños y Lomelí~\cite{Lomeli2018}
examinan el número cromático de la gráfica de disyunción de una familia de puntos
conocida como la doble cadena convexa; al igual que la posición convexa la doble cadena
solamente tiene un dibujo en el plano. El resultado al que se llega en este artículo es que
el anti-thickness de la doble cadena convexa con $k$ puntos en la cadena convexa superior
y $l \geq k$ puntos en la cadea convexa inferior es:
 \[k+l-\Bigl\lfloor \sqrt{2l+\frac{1}{4}} - \frac{1}{2} \Bigr\rfloor\]

 Como es mencionado en~\cite{Dujmovic2017}, encontrar el anti-thickness de gráficas
 completas en posición general es un problema abierto y la mejor cota superior
 conocida es $ n - \lfloor \sqrt{2n + \frac{1}{4}} - \frac{1}{2} \rfloor $
la cual se sigue del caso convexo y la mejor cota inferior conocida es solamente
$\lfloor\frac{n-1}{2}\rfloor$ que se sigue del hecho de que una gráfica de $n$ vértices
con anti-thickness geométrico $k$ tiene a lo más $kn$ aristas.
