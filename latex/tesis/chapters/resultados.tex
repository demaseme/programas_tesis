\section{Cota inferior del anti-thickness geométrico de $K_n$.}
El principal resultado de este trabajo es que encontramos el valor exacto del
anti-thickness geométrico para la gráfica completa de hasta diez vértices en posición general.
Dicho valor exacto fue obtenido mejorando la cota inferior mencionada en el estado del arte
y usando la cota superior actual. Empezaremos explicando el proceso para encontrar la cota inferior.

Un conjunto de $n$ puntos en el plano induce una gráfica completa de $n$ vértices
a la que denotamos $K_n$. Como es completa $|E(K_n)|= \binom{n}{2}$. Cuando se quiere
encontrar una descomposición en thrackles es intuitivo que los thrackles tengan el
mayor número posible de aristas. El siguiente teorema es útil para nosotros ya que
establece el número máximo de aristas que puede tener un thrackle (geométrico).
\begin{theorem}(\cite{Pach2013b})Toda gráfica geométrica de $n$ vértices en la
  que no existen dos aristas disjuntas tiene a lo más $n$ aristas. Esto se cumple
  para toda $n>2$.
\end{theorem}
Como se mencionó en el capítulo de antecedentes, llamamos a estos thrackles máximos.

Toda descomposición de la gráfica completa cubre sus aristas. Si suponemos que existen $k$
thrackles máximos en la descomposición entonces la siguiente desigualdad expresa el número de
thrackles máximos son necesarios para cubrir las $\binom{n}{2}$ aristas de la gráfica completa:
\[ kn \geq \binom{n}{2}. \]

Como los thrackles de la descomposición son geométricos, si buscamos la $k$ más pequeña
para la cual se cumple la desigualdad entonces $k$ es el anti-thickness geométrico de la gráfica completa.

Resolviendo para $k$ la desigualdad anterior tenemos que necesitamos al menos
$k=\left\lceil\frac{n-1}{2}\right\rceil$ thrackles máximos para dar una descomposición
de $K_n$. En otras palabras \[At_g(K_n) \geq  \left\lceil\frac{n-1}{2}\right\rceil.\]

La tabla~\ref{table:attrivialinf} ilustra el valor de la cota inferior del anti-thickness
geométrico obtenida anteriormente para $K_n$ con $n\leq 10$.
\begin{table}[t]
  \centering
  \begin{tabular}{|c|c|}
    \hline
    $n$ & $\left\lceil\frac{n-1}{2}\right\rceil$ \\[5pt] \hline\hline
    3   & 1  \\
    4   & 2  \\
    5   & 2  \\
    6   & 3  \\
    7   & 3  \\
    8   & 4  \\
    9   & 4  \\
    10  & 5  \\ \hline
  \end{tabular}
  \caption{ Valor de la cota inferior del anti-thickness geométrico usando la cota trivial. }
  \label{table:attrivialinf}
\end{table}
Esta cota es la más inmediata ya que usa el hecho de que cada thrackle máximo tiene a lo
sumo tantas aristas como vértices. También es la cota actual para el anti-thickness geométrico.

En el trabajo de~\cite{Fabila-Monroy2018} encuentran que dados dos thrackles
máximos en posición convexa estos comparten una arista y esto se cumple para
cada par de thrackles de la descomposición.
En este trabajo verificamos que este resultado también es válido para conjuntos de
hasta diez puntos en posición general. Usando los tipos de orden de los conjuntos
con hasta diez puntos inducimos la gráfica completa, luego buscamos, para
cada uno de los tipos de orden, todos los thrackles máximos y
finlamente comparamos dichos thrackles a pares y encontramos que:
\begin{itemize}
  \item No existe, para ningún tipo de orden con más de dos thrackles máximos, dos thrackles máximos disjuntos.
  \item Existen tipos de orden con solo un thrackle máximo.
  \item Existen tipos de orden en los que no hay thrackles máximos.
\end{itemize}

Para buscar los thrackles máximos usamos el algoritmo X descrito en la sección Y.
Para comparar los thrackles a pares utilizamos el algoritmo XX descrito también en la sección Y.
El número de los tipos de orden para los cuales existe solamente un thrackle máximo está
en el apéndice XX, la información también puede ser descargada de la siguiente liga: HTTPXX.
El número de los tipos de orden para los cuales no existe un thrackle máximo está en la
siguiente liga: HTTPXX.

Gracias a estos resultados podemos derivar el siguiente lema.
\begin{lemma}\label{lema:thdisjuntos}
  Sean $T_1$ y $T_2$ thrakles máximos en posición general con $|V(T_1)|\leq 10$
  y $|V(T_2)|\leq 10$. $T_1$ y $T_2$ tienen al menos una arista en común.
\end{lemma}
\begin{proof}
  Esto fue probado computacionalmente verificando a pares los thrackles
  encontrados para cada tipo de orden de cada conjunto de hasta diez puntos.
  Para todos los tipos de orden donde existen más de dos thrackles
  máximos encontramos que comparten al menos una arista. Para comparar
  si dos thrackles $T_i,$ y $T_j$ comparten una arista se realiza la intersección
  del conjunto de aristas de los thrackles. Si esta es no vacía entonces hay al menos una arista
  que está en $T_i$ y $T_j$.
\end{proof}
Esto implica que para $n\leq 10$ no es posible encontrar una descomposición
de $\left\lceil\frac{n-1}{2}\right\rceil$ thrackles máximos ya que los thrackles
máximos no son disjuntos a pares y esto infringe las condiciones de una descomposición.
Esto significa que una descomposición por thrackles solo podría contener un thrackle máximo.

Sin embargo es posible encontrar una descomposición a partir de una colección de thrackles
máximos y contar cuántas aristas es posible cubrir con dicha descomposición. Describimos
este proceso en el siguiente lema:
% Por consecuencia es posible contar cuántas aristas son cubiertas
% por una descomposición de thrackles máximos cuyo tamaño es $\left\lceil\frac{n-1}{2}\right\rceil$.
\begin{lemma}\label{lema:existedescomp}
  Sea $\mathcal{C}=\{T_1,T_2,\dots,T_{\left\lceil\frac{n-1}{2}\right\rceil}\}$ una colección
  de thrackles máximos de $K_n$, donde $\bigcup\mathcal{C}$ induce a $K_n$ y $|E(T_i)\cap E(T_j)| = 1$.
  Existe una descomposición $\mathcal{D}$ de $K_n$ inducida por $\mathcal{C}$.
\end{lemma}
\begin{proof}
  % Sea $\mathcal{D}=\emptyset$. Agregar el thrackle $T_1$ a $D$, luego agregar el thrackle compuesto de las aristas de $T_2$
  % excepto aquella que comparte con $T_1$, este thrackle tiene $n-1$ aristas. De manera similar agregar el thrackle compuesto
  % de las aristas de $T_3$ excepto aquella que comparte con $T_1$ y la que comparte con $T_2$, este thrackle tiene $n-2$ aristas.
  % Si se continua agregando los thrackles restantes de $\mathcal{C}$ de la misma manera entonces $\mathcal{D}$ es una descomposición de $K_n$
  % ya que por construcción cada arista de $K_n$ está en exactamente un thrackle de $D$.
  Sea \[T'_i = \{\text{El thrackle inducido por } \{E(T_i) - \bigcup_{k=1}^{i-1} E(T_i)\cap E(T_k)\} \}\]
  es decir $T'_i$ es el thrackle que tiene todas las aristas de $T_i$
  a excepción de aquellas que $T_i$ comparte con $T_{i-1},T_{i-2},\dots,T_{1}$.

  Si $\mathcal{D}=\{T'_1,T'_2,\dots,T'_{\left\lceil\frac{n-1}{2}\right\rceil}\}$ por construcción todas las aristas de $K_n$ están en
  exactamente un elemento de $\mathcal{D}$. Por lo tanto $\mathcal{D}$ es una descomposición de $K_n$.
\end{proof}
Ahora procedemos a contar el número de aristas que pueden ser cubiertas por una descomposición
inducida por una colección de thrackles máximos.
\begin{lemma}\label{lema:numaristascubiertas}
  Sea \[T'_i = \{\text{El thrackle inducido por } \{E(T_i) - \bigcup_{k=1}^{i-1} E(T_i)\cap E(T_k)\} \}\]
  y sea  $\mathcal{D}=\{T'_1,T'_2,\dots,T'_{\left\lceil\frac{n-1}{2}\right\rceil}\}.$

  El número de aristas cubiertas por $\mathcal{D}$ es exactamente :
   \[\displaystyle \sum^n_{i=\left(n-\left\lceil\frac{n-1}{2}\right\rceil\right) + 1}i\]
 \end{lemma}
 \begin{proof}
   Por construcción: $T'_1$ tiene $n$ aristas, luego $T'_2$ tiene $n-1$ aristas, en general $T'_i$ tiene
   $n-(i-1) = n-i+1$ aristas. Como $\mathcal{D}$ tiene $\left\lceil\frac{n-1}{2}\right\rceil$ thrackles
   el número de aristas cubiertas por $\mathcal{D}$ es \[ n + (n-1) + (n-2) + \cdots + (n-\left\lceil\frac{n-1}{2}\right\rceil) +
   (n - \left\lceil\frac{n-1}{2}\right\rceil + 1)\]
   Podemos escribir esta suma como:
   \[\displaystyle \sum^n_{i=\left(n-\left\lceil\frac{n-1}{2}\right\rceil\right) + 1}i\]
 \end{proof}
% Por consecuencia es posible contar cuántas aristas serán cubiertas si la descomposición es de thrackles
% máximos y tiene tamaño $\left\lceil\frac{n-1}{2}\right\rceil$ como sigue: el primer thrackle
% cubre $n$ aristas, el siguiente $n-1$ aristas, el próximo $n-2$ aristas
% y así sucesivamente tantas veces como thrackles tenga la descomposición, en este caso
% existen $\left\lceil\frac{n-1}{2}\right\rceil$ thrackles máximos en la descomposición.
%
% Entonces el número de aristas cubiertas es:
% \[\displaystyle \sum^n_{i=\left(n-\left\lceil\frac{n-1}{2}\right\rceil\right) + 1}i.\]

Con los lemas anteriores es posible probar que la cota inferior del anti-thickness geométrico
de $K_n$ no es justa para toda $n$. El siguiente teorema establece esta afirmación.
\begin{theorem}\label{teo:cotainf}
Sea $\mathcal{D}=\{T'_1,T'_2,\dots,T'_{\left\lceil\frac{n-1}{2}\right\rceil}\}$ una descomposición
de la gráfica completa inducida por una colección de $\left\lceil\frac{n-1}{2}\right\rceil$ thrackles máximos.
$\left\lceil\frac{n-1}{2}\right\rceil$ thrackles máximos no son
suficientes para inducir una descomposición de $K_n$ para toda $3 \leq n \leq 10$.
\end{theorem}
\begin{proof}
  La cota inferior solamente es justa para $n\in\{3,4,5\}$. Mientras que para $n\in \{5,7,8,9,10\}$
  no es posible cubrir todas las aristas usando una descomposición inducida por una colección
  de thrackles máximos.
\end{proof}
En la tabla~\ref{table:attrivialtight} mostramos los casos para los que la cota
inferior del anti-thickness geométrico no es justa usando el lema~\ref{lema:numaristascubiertas}.
Es importante notar que este resultado es válido solamente cuando el dibujo de $K_n$ tiene al menos
$\left\lceil\frac{n-1}{2}\right\rceil$ thrackles máximos.

\begin{table}[t]
  \centering
  \begin{tabular}{|c|c|c|c|}
    \hline
    $n$ & $\left\lceil\frac{n-1}{2}\right\rceil$ & $\sum^n_{i=\left(n-\left\lceil\frac{n-1}{2}\right\rceil\right) + 1}i$ & $\binom{n}{2}$\\[5pt] \hline\hline
    3   & 1  & 3 & 3 \\ \hline
    4   & 2  & 7 & 6 \\ \hline
    5   & 2  & \cellcolor{red!25}9 & 10 \\ \hline
    6   & 3  & 15 & 15 \\ \hline
    7   & 3  & \cellcolor{red!25}18 & 21 \\ \hline
    8   & 4  & \cellcolor{red!25}26 & 28 \\ \hline
    9   & 4  & \cellcolor{red!25}30 & 36 \\ \hline
    10  & 5  & \cellcolor{red!25}40 & 45 \\ \hline
  \end{tabular}
  \caption{ Mostramos cuántas aristas son cubiertas con una descomposición de $\left\lceil\frac{n-1}{2}\right\rceil$  thrackles
  máximos. Se rellenan los casos en los que la descomposición no es válida ya que no cubre todas las aristas. }
  \label{table:attrivialtight}
\end{table}

Gracias al lema~\ref{lema:numaristascubiertas} y al teorema~\ref{teo:cotainf} podemos
dar el valor exacto del número de thrackles máximos necesarios para inducir una
descomposición por thrackles de $K_n$ con $ 3\leq n \leq 10$.

\begin{theorem}\label{teo:nuevacotainf}
  Si $\mathcal{D}$ una descomposición de $K_n$ inducida por $k$ thrackles máximos
  entonces :
  \[
    k = \left\{ \begin{array}{lr}
      \left\lceil\frac{n-1}{2}\right\rceil,     & \text{para } n=\{3,4,6\}\\
      \left\lceil\frac{n-1}{2}\right\rceil + 1, & \text{para } n=\{5,7,8,10\}\\
      \left\lceil\frac{n-1}{2}\right\rceil + 2, & \text{para } n=9
    \end{array} \right\}
  \]
\end{theorem}
\begin{proof}
  La prueba consiste en 
\end{proof}
%
% Finalmente, usando el lema~\ref{lema:thdisjuntos} podemos ajustar la cota inferior.
%
% existe una descomposición de $k$ thrackles máximos entonces la descomposición
% cubre $kn - \binom{k}{2}$ aristas. Esto es exactamente igual que en el caso convexo.
% Podemos decir que para $n\leq 10$: \[ At_g(K_n) \geq  n - \Bigl\lfloor \sqrt{2n + \frac{1}{4}} - \frac{1}{2} \Bigr\rfloor. \]

% Podemos observar que esta cota inferior no es justa para toda $n\leq 10$. Por ejemplo
% si tomamos $n=5$, sabemos que la gráfica completa $K_5$ tiene $\binom{n}{2}=10$ aristas.
% Sabemos también que en $K_5$ los thrackles máximos tienen a lo sumo cinco aristas.
% Si es posible usar dos thrackles máximos para cubrir las diez aristas de $K_5$ significa que
% cada uno aporta exactamente cinco aristas y por lo tanto estos dos thrackles son disjuntos.
% Sin embargo nuestros resultados muestran que no existen dos thrackles máximos disjuntos
% para ningún dibujo de $K_5$. Entonces un thrackle máximo aporta cinco aristas y otro
% thrackle máximo aporta cuatro aristas, esto quiere decir que con dos thrackles máximos
% podemos cubrir nueve de las diez aristas de $K_5$, luego necesitamos un thrackle más
% para cubrir la arista restante. Decimos que el anti-thickness geométrico de $K_5$
% es al menos tres.



\section{Descomposiciones por thrackles máximos.}

En este capítulo reportaremos el pseudocódigo de algoritmos usados durante el desarrollo
del proyecto así como los resultados que obtuvimos con dichos algoritmos.
Asímismo presentamos la prueba de que un thrackle que no es máximo no siempre
puede ser completado a uno máximo en posición general, mientras que en posición
convexa sí es posible.

% Inspirado en el método usado por~\cite{Fabila-Monroy2018} para posición convexa
% decidimos buscar para $n \leq 10$ descomposiciones de $K_n$ en la que los thrackles
% sean todos maximos y se alcance el anti-thickness establecido por el tipo de orden
% en posición convexa. Para esto buscamos para cada tipo de orden de cada $n \leq 10$
% todos los thrackles máximos y evaluamos cada combinación de tamaño $k_n$ con
% $k_n$ el anti-thickness convexo para conjuntos de tamaño $n$.
En el trabajo de~\cite{Fabila-Monroy2018} se encuentran descomposiciones de gráficas
geométricas en posición convexa usando thrackles máximos que comparten aristas a pares.
Usando esta idea y los datos de la tabla \ref{table:atconvexo} decidimos
buscar descomposiciones de $K_n$ para $n \leq 10$ en las que los thrackles
son todos máximos y cuyo tamaño sea el mismo del anti-thickness convexo para $K_n$.

\begin{table}[t]
  \centering
  \begin{tabular}{|c|c|}
    \hline
    $n$ & Anti-thickness convexo de $K_n$ ($At_c(n)$) \\ \hline\hline
    3   & 1  \\
    4   & 2  \\
    5   & 3  \\
    6   & 3  \\
    7   & 4  \\
    8   & 5  \\
    9   & 6  \\
    10  & 6  \\ \hline
  \end{tabular}
  \caption{ Anti-thickness convexo para $n\leq 10$ basado en el resultado de ~\cite{Fabila-Monroy2018}.}
  \label{table:atconvexo}
\end{table}

En resumen: para cada $n$ tomamos las combinaciones de $cat(n)$ thrackles máximos y
verificamos si alguna de estas combinaciones es una descomposición de $K_n$. Este proceso
lo repetimos para cada uno de los tipos de orden que hay para cada $n$.
Encontramos que sí existen tipos de orden que no corresponden al de posición convexa que también
alcanzan el anti-thickness convexo. Los resultados se muestran en la tabla \ref{table:res_desc_th_max}.
\begin{table}[t]
  \centering
  \begin{tabular}{|c|c|c|}
    \hline
    $n$   & Tipo de Orden & $k_n$ \\ \hline\hline
    8 & 12   & 5  \\
    8 & 54   & 5  \\ \hline
    9 & 12   & 6  \\
    9 & 52   & 6  \\
    9 & 54   & 6  \\
    9 & 80   & 6  \\
    9 & 696  & 6  \\
    9 & 1080 & 6  \\
    9 & 1287 & 6  \\ \hline
   10 & 81   & 6  \\
   10 & 1328 & 6  \\
   10 & 2243 & 6  \\ \hline
  \end{tabular}

  \caption{Tipos de orden para los que existe al menos una descomposición en thrackles máximos
  de tamaño igual al anti-thickness del tipo de orden convexo.}
  \label{table:res_desc_th_max}
\end{table}

Para los casos de $n \in \{3,4,\dots,7\}$ no pudimos encontrar una descomposición que cumpliera
con las caracteristicas antes descritas, esto es porque existen pocos tipos de orden
cuya unión de thrackles máximos cubran las aristas de la gráfica completa.
Los resultados fueron obtenidos con un algoritmo que primero evalúa
cuáles son los tipos de orden que podrían tener una descomposición por thrackles máximos;
se seleccionan aquellos que tengan suficientes thrackles máximos para cubrir todas las
aristas de la gráfica completa, en otras palabras que la unión de los thrackles
máximos en determinado tipo de orden cubran las $\binom{n}{2}$ aristas. Los pseudocódigos
de los algoritmos usados se encuentra en el algoritmo \ref{algo:maxthrackledecom} y
el algoritmo \ref{algo:finddecompositions}.

\begin{algorithm}[h!]
  \begin{algorithmic}[1]
    \Procedure{MaxThrackleDecom}{$n$}
    \State $vectorOT \gets \textit{valid-thrackles()}$
    \State $k \gets convexAt(n)$
    \ForEach {$ot \in vectorOT $}
        \State $n_{thr} \gets \textit{number of max thrackles on order type ot}$
        \State \textit{find-all-decomposition-of-size($n_{thr}$,k)}
    \EndFor
    \EndProcedure
  \end{algorithmic}
  \caption{Pseudocódigo del algoritmo que encuentra descomposiciones por thrackles
  máximos para todos los tipos de orden de una $n$ dada.}
  \label{algo:maxthrackledecom}
\end{algorithm}

\begin{algorithm}[h!]
  \begin{algorithmic}[1]
    \Procedure{find-all-decomposition-of-size}{$n_{thr},k$}
    \While{There is a combination $c$ of size $k$ from $\{0,1,\dots,n_{thr}\}$}
      \If{$c$ is a decomposition}
        \State visit $c$
      \EndIf
    \EndWhile
    \EndProcedure
    \caption{Pseudocódigo del algoritmo que encuentra combinaciones de $k$ thrackles
    máximos, si la combinación es una descomposición se visita.}
    \label{algo:finddecompositions}
  \end{algorithmic}
\end{algorithm}

Ejecutamos la implementación del algoritmo en el cluster:
para $n=8$ el resultado es obtenido en menos de un segundo mientras que para
$n=9$ y $n=10$ se necesitaron al rededor de 1 día y 6 días respectivamente.
Las decomposiciones encontradas pueden verse con más detalle en el apéndice XXXXX.

En el desarrollo del trabajo nos preguntamos por qué existen
otros tipos de orden diferente del convexo que tienen el mismo anti-thickness.
Algo en lo que pensamos fue en analizar de alguna manera la estructura de los
thrackles en dichos tipos de orden y por ello calculamos, para las descomposiciones
obtenidas mediante el método anteriormente descrito, el número de cruce de cada
uno de los thrackles de la descomposición. Se observa que en la mayoría de los
casos la mitad de los thrackles de las descomposiciones
tienen el número de cruce mínimo para $n$ vértices y la otra mitad es más
cercano al mayor número de cruce para $n$ vértices. Estos resultados pueden
estudiarse con más detalle een el apéndice XXXXXX.

\section{Anti-thickness geométrico exacto de $K_n$ con $3\leq n \leq 9$.}
\chaptermark{$gat(n)$ exacto para $3\leq n \leq 9$.}
Dado que la cota superior está dada por el anti-thickness convexo decidimos tratar
de ajustar la cota inferior ya que creemos que el anti-thickness geométrico es igual
al anti-thickness convexo. Un enfoque para ajustar la cota inferior es obtener el
anti-thickness de cada dibujo de $K_n$, esto es, obtener el anti-thickness de cada
tipo de orden para $K_n$ y seleccionar el menor de todos. Sin embargo, el algoritmo
exhaustivo para encontrar el anti-thickness tarda al rededor de 7 horas para un solo
tipo de orden cuando $n=8$, si para $n=8$ hay 3315 tipos de orden requeririamos
cerca de 960 días para acabar dicha tarea.

Por esta razón decidimos analizar la estructura de las posibles descomposiciones;
como $K_n$ tiene $\binom{n}{2}$ aristas y los thrackles de la descomposición deben
cubrirlas todas podemos buscar particiones de enteros de la forma $a_1 + a_2 + \dots +
a_k = \binom{n}{2}$.

A manera de ejemplo, mostraremos como se ajusta la cota inferior del anti-thickness
geométrico para $K_5$. En $K_5$ existen 10 aristas. Las siguientes son
particiones del entero 10:
\[
\begin{array}{l l}
  1 + 1 + 1 + 1 + 1 + 1 + 1 + 1 + 1 + 1 & 2 + 1 + 1 + 1 + 1 + 1 + 1 + 1 + 1\\
  3 + 1 + 1 + 1 + 1 + 1 + 1 + 1         & 2 + 2 + 1 + 1 + 1 + 1 + 1 + 1    \\
  4 + 1 + 1 + 1 + 1 + 1 + 1             & 3 + 2 + 1 + 1 + 1 + 1 + 1        \\
  2 + 2 + 2 + 1 + 1 + 1 + 1             & 5 + 1 + 1 + 1 + 1 + 1            \\
  4 + 2 + 1 + 1 + 1 + 1                 & 3 + 3 + 1 + 1 + 1 + 1            \\
  3 + 2 + 2 + 1 + 1 + 1                 & 2 + 2 + 2 + 2 + 1 + 1            \\
  6 + 1 + 1 + 1 + 1                     & 5 + 2 + 1 + 1 + 1                \\
  4 + 3 + 1 + 1 + 1                     & 4 + 2 + 2 + 1 + 1                \\
  3 + 3 + 2 + 1 + 1                     & 3 + 2 + 2 + 2 + 1                \\
  2 + 2 + 2 + 2 + 2                     & 7 + 1 + 1 + 1                    \\
  6 + 2 + 1 + 1                         & 5 + 3 + 1 + 1                    \\
  5 + 2 + 2 + 1                         & 4 + 4 + 1 + 1                    \\
  4 + 3 + 2 + 1                         & 4 + 2 + 2 + 2                    \\
  3 + 3 + 3 + 1                         & 3 + 3 + 2 + 2                    \\
  8 + 1 + 1                             & 7 + 2 + 1                        \\
  6 + 3 + 1                             & 6 + 2 + 2                        \\
  5 + 4 + 1                             & 5 + 3 + 2                        \\
  4 + 4 + 2                             & 4 + 3 + 3                        \\
  9 + 1                                 & 8 + 2                            \\
  7 + 3                                 & 6 + 4                            \\
  5 + 5
\end{array}
\]
Ahora bien, algunas de estas particiones pueden ser usadas como guía para encontrar
una descomposición en thrackles para $K_5$. Si tomamos, por ejemplo, la partición $5+4+1$
estaríamos buscando una descomposición por 3 thrackles: uno de tamaño 5, uno de tamaño 4
y otro de tamaño 1. Es importante notar que como las particiones de un entero $k$
suman exactamente $k$, los thrackles de la descomposición tienen que ser disjuntos en aristas
cuando los tamaños corresponden a los enteros de la partición de $k$.
La partición $5+4+1$ podría ser posible de encontrar, sin embargo, podemos
deshacernos de ciertas particiones que estamos seguros jamás encontraremos como son
aquellas particiones que tienen un entero mayor a $5$ puesto que para un conjunto
de $5$ vértices el thrackle geométrico más grande tiene $5$ aristas, esto también se cumple
para todo $n$. Desaparecerían entonces particiones como $7+2+1$ o $9+1$ por mencionar algunas.

Nuestro conjunto de particiones posibles se ve ahora de la siguiente manera:
\[
\begin{array}{l l}
  1 + 1 + 1 + 1 + 1 + 1 + 1 + 1 + 1 + 1 & 2 + 1 + 1 + 1 + 1 + 1 + 1 + 1 + 1\\
  3 + 1 + 1 + 1 + 1 + 1 + 1 + 1         & 2 + 2 + 1 + 1 + 1 + 1 + 1 + 1    \\
  4 + 1 + 1 + 1 + 1 + 1 + 1             & 3 + 2 + 1 + 1 + 1 + 1 + 1        \\
  2 + 2 + 2 + 1 + 1 + 1 + 1             & 5 + 1 + 1 + 1 + 1 + 1            \\
  4 + 2 + 1 + 1 + 1 + 1                 & 3 + 3 + 1 + 1 + 1 + 1            \\
  3 + 2 + 2 + 1 + 1 + 1                 & 2 + 2 + 2 + 2 + 1 + 1            \\
  5 + 2 + 1 + 1 + 1                     & 4 + 3 + 1 + 1 + 1                \\
  4 + 2 + 2 + 1 + 1                     & 3 + 3 + 2 + 1 + 1                \\
  3 + 2 + 2 + 2 + 1                     & 2 + 2 + 2 + 2 + 2                \\
  5 + 3 + 1 + 1                         & 5 + 2 + 2 + 1                    \\
  4 + 3 + 2 + 1                         & 4 + 4 + 1 + 1                    \\
  3 + 3 + 3 + 1                         & 4 + 2 + 2 + 2                    \\
  5 + 4 + 1                             & 3 + 3 + 2 + 2                    \\
  4 + 4 + 2                             & 5 + 3 + 2                        \\
  5 + 5                                 & 4 + 3 + 3
\end{array}
\]

Sin embargo, como buscamos ajustar la cota inferior del anti-thickness no nos
interesa encontrar descomposiciones cuyo tamaño sea mayor a la cota superior
del anti-thickness dada por $n - \lfloor \sqrt{2n + 1/4} - 1/2 \rfloor$, en el
caso de $n=5$, evitaremos buscar descomposiciones con un tamaño mayor a 3. Dejando
así las siguientes particiones disponibles:
\[
\begin{array}{l l}
  5 + 4 + 1                 & 4 + 4 + 2                                 \\
  4 + 3 + 3                 & 5 + 3 + 2                                 \\
  5 + 5
\end{array}
\]

Finalmente, vamos a remover las particiones cuyo tamaño sea igual al anti-thickness
convexo de $K_5$, esto porque sabemos que en efecto la posición convexa otorga
descomposiciones de ese tamaño. Esto nos deja con una única partición posible :
\[ 5 + 5 \] Esto significa que debemos averiguar si existe una descomposición
de $K_5$ por dos thrackles de tamaño 5, en este caso dos thrackles máximos. No obstante,
al buscar las thrackles máximos para todos los tipos de orden de $K_5$ encontramos
que no existen dos thrackles máximos que sean disjuntos en aristas, por esto no
es posible dar una descomposición de $K_5$ en dos thrackles máximos. Y luego,
el anti-thickness de $K_5$ es mayor a 2. Como la cota superior del anti-thickness
de $K_5$ es 3 podemos decir que el anti-thickness geométrico de $K_5$ es exactamente 3.

De esta manera podemos acotar el anti-thickness geométrico de $K_n$: examinar particiones
del entero $\binom{n}{2}$ con las siguientes condiciones:
\begin{itemize}
  \item La longitud de la partición es menor que el anti-thickness convexo de $K_n$.
  \item Solo existe una ocurrencia del entero $n$ en la partición.
\end{itemize}

Siguiendo las condiciones anteriores buscamos las particiones válidas para $K_n$
con $n\in[3,9]$. Encontramos que para $n\in[3,7]$ no existen particiones que
cumplan las condiciones, por lo que podemos decir que el anti-thickness geométrico
de $K_n$ para $n\in[3,7]$ es igual al anti-thickness convexo.

Para $K_8$ encontramos las siguientes particiones válidas:

\[
\begin{array}{l l}
8 + 7 + 7 + 6 & 7 + 7 + 7 + 7
\end{array}
\]

No fue posible encontrar una descomposición en thrackles usando alguna de estas
particiones, por lo que podemos decir que $K_8$ tiene anti-thickness geométrico
mayor a 4 y luego el anti-thickness geométrico de $K_8$ es exactamente 5.

Por otro lado para ajustar la cota inferior del anti-thickness de $K_9$,
tenemos las siguientes particiones válidas:
\[
\begin{array}{l l}
9 + 8 + 8 + 8 + 3 & 9 + 8 + 8 + 7 + 4 \\
9 + 8 + 8 + 6 + 5 & 9 + 8 + 7 + 7 + 5 \\
9 + 8 + 7 + 6 + 6 & 9 + 7 + 7 + 7 + 6 \\
8 + 8 + 8 + 8 + 4 & 8 + 8 + 8 + 7 + 5 \\
8 + 8 + 8 + 6 + 6 & 8 + 8 + 7 + 7 + 6 \\
8 + 7 + 7 + 7 + 7
\end{array}
\]

Para cada una de las particiones se diseñó un algoritmo que evalúa todos los thrackles
de tamaño 9, 8, 7 y 6. Los resultados fueron los siguientes:

\begin{itemize}
  \item 9 + 8 + 8 + 8 + 3 - Probado con 9+8+8.
  \item 9 + 8 + 8 + 7 + 4 - Probado con 9+8+8.
  \item 9 + 8 + 8 + 6 + 5 - Probado con 9+8+8. 150000ms
  \item 9 + 8 + 7 + 7 + 5 - Probado con 9+8+7+6+6
  \item 9 + 8 + 7 + 6 + 6 - Probado con 9+8+7+6+6. 981709 ms.
  \item 9 + 7 + 7 + 7 + 6 - Probado con 9+7+7+7+6. 2.11354e+06 ms.
  \item 8 + 8 + 8 + 8 + 4 - Probado con 8+8+8+8. 300888 ms.
  \item 8 + 8 + 8 + 7 + 5 - Probado con 8+8+8+6+6.
  \item 8 + 8 + 8 + 6 + 6 - Probado con 8+8+8+6+6. 569735 ms.
  \item 8 + 8 + 7 + 7 + 6 - Probado con 8+8+7+7+6. 6.39485e+06 - 1 Hora, 46 minutos.
  \item 8 + 7 + 7 + 7 + 7 - Probado con 8+7+7+7+7. 1.23716e+08 - 34 Horas, 21 minutos.
\end{itemize}

En la mayoría de los casos no fue necesario examinar toda la partición, por ejemplo
para la partición $9+8+8+8+3$, encontramos que no hay 3 thrackles, para ningún
tipo de orden diferente del convexo, donde uno sea de tamaño 9
y los otros dos de tamaño 8 que sean disjuntos en aristas y por esta razón no es necesario
seguir examinando la partición a fondo.

Como para ninguna partición fue posible encontrar una descomposición en thrackles
podemos decir que el anti-thickness geométrico de $K_9$ es mayor a 5. Y como la cota
superior del anti-thickness geométrico es 6 decimos que el anti-thickness de $K_9$
es exactamente 6.
