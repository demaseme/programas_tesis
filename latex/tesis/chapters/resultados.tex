\section{Descomposiciones por thrackles máximos.}
Inspirado en el método usado por~\cite{Fabila-Monroy2018} para posición convexa
decidimos buscar para $n \leq 10$ descomposiciones de $K_n$ en la que los thrackles
sean todos maximos y se alcance el anti-thickness establecido por el tipo de orden
en posición convexa. Para esto buscamos para cada tipo de orden de cada $n \leq 10$
todos los thrackles máximos y evaluamos cada combinación de tamaño $k_n$ con
$k_n$ el anti-thickness convexo para conjuntos de tamaño $n$. Encontramos que
sí existen tipos de orden que no corresponden al de posición convexa que también
alcanzan el anti-thickness convexo. Los resultados se muestran en la tabla \ref{table:res_desc_th_max}.
\begin{table}[t]
  \centering
  \begin{tabular}{|c|c|c|}
    \hline
    $n$   & Tipo de Orden & $k_n$ \\ \hline\hline
    8 & 12   & 5  \\
    8 & 54   & 5  \\ \hline
    9 & 12   & 6  \\
    9 & 52   & 6  \\
    9 & 54   & 6  \\
    9 & 80   & 6  \\
    9 & 696  & 6  \\
    9 & 1080 & 6  \\
    9 & 1287 & 6  \\ \hline
   10 & 81   & 6  \\
   10 & 1328 & 6  \\
   10 & 2243 & 6  \\ \hline
  \end{tabular}

  \caption{Tipos de orden para los que existe al menos una descomposición en thrackles máximos
  de tamaño igual al anti-thickness del tipo de orden convexo.}
  \label{table:res_desc_th_max}
\end{table}

Para los casos de $n \in [3,7]$ no existen tipos de orden que puedan ser descompuestos
de esta manera. Los resultados fueron obtenidos con un algoritmo que primero evalúa
cuáles son los tipos de orden que podrían tener una descomposición por thrackles máximos;
se seleccionan aquellos que tengan suficientes thrackles máximos para cubrir todas las
aristas de la gráfica completa, en otras palabras que la unión de los thrackles
máximos en determinado tipo de orden cubran las $\binom{n}{2}$ aristas.

El algoritmo fue ejecutado en el cluster, para $n=8$ el resultado es obtenido en menos
de un segundo mientras que para $n=9$ y $n=10$ se necesitaron al rededor de
1 día y 6 días respectivamente. Las decomposiciones encontradas pueden verse con
más detalle en el apéndice XXXXX.

También se calculó, para las descomposiciones obtenidas mediante este método, el
número de cruce de cada uno de los thrackles de la descomposición. Se observa que
en la mayoría de los casos la mitad de los thrackles de las descomposiciones
tienen el número de cruce mínimo para $n$ vértices y la otra mitad es más
cercano al mayor número de cruce para $n$ vértices.

\section{Ajustando la cota inferior para $K_n$ con $n\leq 9$}

Dado que la cota superior está dada por el anti-thickness convexo decidimos tratar
de ajustar la cota inferior ya que creemos que el anti-thickness geométrico es igual
al anti-thickness convexo. Un enfoque para ajustar la cota inferior es obtener el
anti-thickness de cada dibujo de $K_n$, esto es, obtener el anti-thickness de cada
tipo de orden para $K_n$ y seleccionar el menor de todos. Sin embargo, el algoritmo
exhaustivo para encontrar el anti-thickness tarda al rededor de 7 horas para un solo
tipo de orden cuando $n=8$, si para $n=8$ hay 3315 tipos de orden requeririamos
cerca de 960 días para acabar dicha tarea.

Por esta razón decidimos analizar la estructura de las posibles descomposiciones;
como $K_n$ tiene $\binom{n}{2}$ aristas y los thrackles de la descomposición deben
cubrirlas todas podemos buscar particiones de enteros de la forma $a_1 + a_2 + \dots +
a_k = \binom{n}{2}$ por ejemplo para $K_5$ existen 10 aristas. Las siguientes son
particiones del entero 10:
