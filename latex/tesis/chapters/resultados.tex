\section{Descomposiciones por thrackles máximos.}

En este capítulo reportaremos el pseudocódigo de algoritmos usados durante el desarrollo
del proyecto así como los resultados que obtuvimos con dichos algoritmos.
Asímismo presentamos la prueba de que un thrackle que no es máximo no siempre
puede ser completado a uno máximo en posición general, mientras que en posición
convexa sí es posible.

% Inspirado en el método usado por~\cite{Fabila-Monroy2018} para posición convexa
% decidimos buscar para $n \leq 10$ descomposiciones de $K_n$ en la que los thrackles
% sean todos maximos y se alcance el anti-thickness establecido por el tipo de orden
% en posición convexa. Para esto buscamos para cada tipo de orden de cada $n \leq 10$
% todos los thrackles máximos y evaluamos cada combinación de tamaño $k_n$ con
% $k_n$ el anti-thickness convexo para conjuntos de tamaño $n$.
En el trabajo de~\cite{Fabila-Monroy2018} se encuentran descomposiciones de gráficas
geométricas en posición convexa usando thrackles máximos que comparten aristas a pares.
Usando esta idea y los datos de la tabla \ref{table:atconvexo} decidimos
buscar descomposiciones de $K_n$ para $n \leq 10$ en las que los thrackles
son todos máximos y cuyo tamaño sea el mismo del anti-thickness convexo para $K_n$.

\begin{table}[t]
  \centering
  \begin{tabular}{|c|c|}
    \hline
    $n$ & Anti-thickness convexo de $K_n$ ($cat(n)$) \\ \hline\hline
    3   & 1  \\
    4   & 2  \\
    5   & 3  \\
    6   & 3  \\
    7   & 4  \\
    8   & 5  \\
    9   & 6  \\
    10  & 6  \\ \hline
  \end{tabular}
  \caption{ Anti-thickness convexo para $n\leq 10$ basado en el resultado de ~\cite{Fabila-Monroy2018}.}
  \label{table:atconvexo}
\end{table}

En resumen: para cada $n$ tomamos las combinaciones de $cat(n)$ thrackles máximos y
verificamos si alguna de estas combinaciones es una descomposición de $K_n$. Este proceso
lo repetimos para cada uno de los tipos de orden que hay para cada $n$.
Encontramos que sí existen tipos de orden que no corresponden al de posición convexa que también
alcanzan el anti-thickness convexo. Los resultados se muestran en la tabla \ref{table:res_desc_th_max}.
\begin{table}[t]
  \centering
  \begin{tabular}{|c|c|c|}
    \hline
    $n$   & Tipo de Orden & $k_n$ \\ \hline\hline
    8 & 12   & 5  \\
    8 & 54   & 5  \\ \hline
    9 & 12   & 6  \\
    9 & 52   & 6  \\
    9 & 54   & 6  \\
    9 & 80   & 6  \\
    9 & 696  & 6  \\
    9 & 1080 & 6  \\
    9 & 1287 & 6  \\ \hline
   10 & 81   & 6  \\
   10 & 1328 & 6  \\
   10 & 2243 & 6  \\ \hline
  \end{tabular}

  \caption{Tipos de orden para los que existe al menos una descomposición en thrackles máximos
  de tamaño igual al anti-thickness del tipo de orden convexo.}
  \label{table:res_desc_th_max}
\end{table}

Para los casos de $n \in \{3,4,\dots,7\}$ no pudimos encontrar una descomposición que cumpliera
con las caracteristicas antes descritas, esto es porque existen pocos tipos de orden
cuya unión de thrackles máximos cubran las aristas de la gráfica completa.
Los resultados fueron obtenidos con un algoritmo que primero evalúa
cuáles son los tipos de orden que podrían tener una descomposición por thrackles máximos;
se seleccionan aquellos que tengan suficientes thrackles máximos para cubrir todas las
aristas de la gráfica completa, en otras palabras que la unión de los thrackles
máximos en determinado tipo de orden cubran las $\binom{n}{2}$ aristas. Los pseudocódigos
de los algoritmos usados se encuentra en el algoritmo \ref{algo:maxthrackledecom} y
el algoritmo \ref{algo:finddecompositions}.

\begin{algorithm}
  \begin{algorithmic}[1]
    \Procedure{MaxThrackleDecom}{$n$}
    \State $vectorOT \gets \textit{valid-thrackles()}$
    \State $k \gets convexAt(n)$
    \ForEach {$ot \in vectorOT $}
        \State $n_{thr} \gets \textit{number of max thrackles on order type ot}$
        \State \textit{find-all-decomposition-of-size($n_{thr}$,k)}
    \EndFor
    \EndProcedure
  \end{algorithmic}
  \caption{Pseudocódigo del algoritmo que encuentra descomposiciones por thrackles
  máximos para todos los tipos de orden de una $n$ dada.}
  \label{algo:maxthrackledecom}
\end{algorithm}

\begin{algorithm}
  \begin{algorithmic}[1]
    \Procedure{find-all-decomposition-of-size}{$n_{thr},k$}
    \While{There is a combination $c$ of size $k$ from $\{0,1,\dots,n_{thr}\}$}
      \If{$c$ is a decomposition}
        \State visit $c$
      \EndIf
    \EndWhile
    \EndProcedure
    \caption{Pseudocódigo del algoritmo que encuentra combinaciones de $k$ thrackles
    máximos, si la combinación es una descomposición se visita.}
    \label{algo:finddecompositions}
  \end{algorithmic}
\end{algorithm}

Ejecutamos la implementación del algoritmo en el cluster:
para $n=8$ el resultado es obtenido en menos de un segundo mientras que para
$n=9$ y $n=10$ se necesitaron al rededor de 1 día y 6 días respectivamente.
Las decomposiciones encontradas pueden verse con más detalle en el apéndice XXXXX.

También se calculó, para las descomposiciones obtenidas mediante este método, el
número de cruce de cada uno de los thrackles de la descomposición. Se observa que
en la mayoría de los casos la mitad de los thrackles de las descomposiciones
tienen el número de cruce mínimo para $n$ vértices y la otra mitad es más
cercano al mayor número de cruce para $n$ vértices.

\section{Ajustando el anti-thickness geométrico de $K_n$ con $n\in[3,9]$.}
Dado que la cota superior está dada por el anti-thickness convexo decidimos tratar
de ajustar la cota inferior ya que creemos que el anti-thickness geométrico es igual
al anti-thickness convexo. Un enfoque para ajustar la cota inferior es obtener el
anti-thickness de cada dibujo de $K_n$, esto es, obtener el anti-thickness de cada
tipo de orden para $K_n$ y seleccionar el menor de todos. Sin embargo, el algoritmo
exhaustivo para encontrar el anti-thickness tarda al rededor de 7 horas para un solo
tipo de orden cuando $n=8$, si para $n=8$ hay 3315 tipos de orden requeririamos
cerca de 960 días para acabar dicha tarea.

Por esta razón decidimos analizar la estructura de las posibles descomposiciones;
como $K_n$ tiene $\binom{n}{2}$ aristas y los thrackles de la descomposición deben
cubrirlas todas podemos buscar particiones de enteros de la forma $a_1 + a_2 + \dots +
a_k = \binom{n}{2}$.

A manera de ejemplo, mostraremos como se ajusta la cota inferior del anti-thickness
geométrico para $K_5$. En $K_5$ existen 10 aristas. Las siguientes son
particiones del entero 10:
\[
\begin{array}{l l}
  1 + 1 + 1 + 1 + 1 + 1 + 1 + 1 + 1 + 1 & 2 + 1 + 1 + 1 + 1 + 1 + 1 + 1 + 1\\
  3 + 1 + 1 + 1 + 1 + 1 + 1 + 1         & 2 + 2 + 1 + 1 + 1 + 1 + 1 + 1    \\
  4 + 1 + 1 + 1 + 1 + 1 + 1             & 3 + 2 + 1 + 1 + 1 + 1 + 1        \\
  2 + 2 + 2 + 1 + 1 + 1 + 1             & 5 + 1 + 1 + 1 + 1 + 1            \\
  4 + 2 + 1 + 1 + 1 + 1                 & 3 + 3 + 1 + 1 + 1 + 1            \\
  3 + 2 + 2 + 1 + 1 + 1                 & 2 + 2 + 2 + 2 + 1 + 1            \\
  6 + 1 + 1 + 1 + 1                     & 5 + 2 + 1 + 1 + 1                \\
  4 + 3 + 1 + 1 + 1                     & 4 + 2 + 2 + 1 + 1                \\
  3 + 3 + 2 + 1 + 1                     & 3 + 2 + 2 + 2 + 1                \\
  2 + 2 + 2 + 2 + 2                     & 7 + 1 + 1 + 1                    \\
  6 + 2 + 1 + 1                         & 5 + 3 + 1 + 1                    \\
  5 + 2 + 2 + 1                         & 4 + 4 + 1 + 1                    \\
  4 + 3 + 2 + 1                         & 4 + 2 + 2 + 2                    \\
  3 + 3 + 3 + 1                         & 3 + 3 + 2 + 2                    \\
  8 + 1 + 1                             & 7 + 2 + 1                        \\
  6 + 3 + 1                             & 6 + 2 + 2                        \\
  5 + 4 + 1                             & 5 + 3 + 2                        \\
  4 + 4 + 2                             & 4 + 3 + 3                        \\
  9 + 1                                 & 8 + 2                            \\
  7 + 3                                 & 6 + 4                            \\
  5 + 5
\end{array}
\]
Ahora bien, algunas de estas particiones pueden ser usadas como guía para encontrar
una descomposición en thrackles para $K_5$. Si tomamos, por ejemplo, la partición $5+4+1$
estaríamos buscando una descomposición por 3 thrackles: uno de tamaño 5, uno de tamaño 4
y otro de tamaño 1. Es importante notar que como las particiones de un entero $k$
suman exactamente $k$, los thrackles de la descomposición tienen que ser disjuntos en aristas
cuando los tamaños corresponden a los enteros de la partición de $k$.
La partición $5+4+1$ podría ser posible de encontrar, sin embargo, podemos
deshacernos de ciertas particiones que estamos seguros jamás encontraremos como son
aquellas particiones que tienen un entero mayor a $5$ puesto que para un conjunto
de $5$ vértices el thrackle geométrico más grande tiene $5$ aristas, esto también se cumple
para todo $n$. Desaparecerían entonces particiones como $7+2+1$ o $9+1$ por mencionar algunas.

Nuestro conjunto de particiones posibles se ve ahora de la siguiente manera:
\[
\begin{array}{l l}
  1 + 1 + 1 + 1 + 1 + 1 + 1 + 1 + 1 + 1 & 2 + 1 + 1 + 1 + 1 + 1 + 1 + 1 + 1\\
  3 + 1 + 1 + 1 + 1 + 1 + 1 + 1         & 2 + 2 + 1 + 1 + 1 + 1 + 1 + 1    \\
  4 + 1 + 1 + 1 + 1 + 1 + 1             & 3 + 2 + 1 + 1 + 1 + 1 + 1        \\
  2 + 2 + 2 + 1 + 1 + 1 + 1             & 5 + 1 + 1 + 1 + 1 + 1            \\
  4 + 2 + 1 + 1 + 1 + 1                 & 3 + 3 + 1 + 1 + 1 + 1            \\
  3 + 2 + 2 + 1 + 1 + 1                 & 2 + 2 + 2 + 2 + 1 + 1            \\
  5 + 2 + 1 + 1 + 1                     & 4 + 3 + 1 + 1 + 1                \\
  4 + 2 + 2 + 1 + 1                     & 3 + 3 + 2 + 1 + 1                \\
  3 + 2 + 2 + 2 + 1                     & 2 + 2 + 2 + 2 + 2                \\
  5 + 3 + 1 + 1                         & 5 + 2 + 2 + 1                    \\
  4 + 3 + 2 + 1                         & 4 + 4 + 1 + 1                    \\
  3 + 3 + 3 + 1                         & 4 + 2 + 2 + 2                    \\
  5 + 4 + 1                             & 3 + 3 + 2 + 2                    \\
  4 + 4 + 2                             & 5 + 3 + 2                        \\
  5 + 5                                 & 4 + 3 + 3
\end{array}
\]

Sin embargo, como buscamos ajustar la cota inferior del anti-thickness no nos
interesa encontrar descomposiciones cuyo tamaño sea mayor a la cota superior
del anti-thickness dada por $n - \lfloor \sqrt{2n + 1/4} - 1/2 \rfloor$, en el
caso de $n=5$, evitaremos buscar descomposiciones con un tamaño mayor a 3. Dejando
así las siguientes particiones disponibles:
\[
\begin{array}{l l}
  5 + 4 + 1                 & 4 + 4 + 2                                 \\
  4 + 3 + 3                 & 5 + 3 + 2                                 \\
  5 + 5
\end{array}
\]

Finalmente, vamos a remover las particiones cuyo tamaño sea igual al anti-thickness
convexo de $K_5$, esto porque sabemos que en efecto la posición convexa otorga
descomposiciones de ese tamaño. Esto nos deja con una única partición posible :
\[ 5 + 5 \] Esto significa que debemos averiguar si existe una descomposición
de $K_5$ por dos thrackles de tamaño 5, en este caso dos thrackles máximos. No obstante,
al buscar las thrackles máximos para todos los tipos de orden de $K_5$ encontramos
que no existen dos thrackles máximos que sean disjuntos en aristas, por esto no
es posible dar una descomposición de $K_5$ en dos thrackles máximos. Y luego,
el anti-thickness de $K_5$ es mayor a 2. Como la cota superior del anti-thickness
de $K_5$ es 3 podemos decir que el anti-thickness geométrico de $K_5$ es exactamente 3.

De esta manera podemos acotar el anti-thickness geométrico de $K_n$: examinar particiones
del entero $\binom{n}{2}$ con las siguientes condiciones:
\begin{itemize}
  \item La longitud de la partición es menor que el anti-thickness convexo de $K_n$.
  \item Solo existe una ocurrencia del entero $n$ en la partición.
\end{itemize}

Siguiendo las condiciones anteriores buscamos las particiones válidas para $K_n$
con $n\in[3,9]$. Encontramos que para $n\in[3,7]$ no existen particiones que
cumplan las condiciones, por lo que podemos decir que el anti-thickness geométrico
de $K_n$ para $n\in[3,7]$ es igual al anti-thickness convexo.

Para $K_8$ encontramos las siguientes particiones válidas:

\[
\begin{array}{l l}
8 + 7 + 7 + 6 & 7 + 7 + 7 + 7
\end{array}
\]

No fue posible encontrar una descomposición en thrackles usando alguna de estas
particiones, por lo que podemos decir que $K_8$ tiene anti-thickness geométrico
mayor a 4 y luego el anti-thickness geométrico de $K_8$ es exactamente 5.

Por otro lado para ajustar la cota inferior del anti-thickness de $K_9$,
tenemos las siguientes particiones válidas:
\[
\begin{array}{l l}
9 + 8 + 8 + 8 + 3 & 9 + 8 + 8 + 7 + 4 \\
9 + 8 + 8 + 6 + 5 & 9 + 8 + 7 + 7 + 5 \\
9 + 8 + 7 + 6 + 6 & 9 + 7 + 7 + 7 + 6 \\
8 + 8 + 8 + 8 + 4 & 8 + 8 + 8 + 7 + 5 \\
8 + 8 + 8 + 6 + 6 & 8 + 8 + 7 + 7 + 6 \\
8 + 7 + 7 + 7 + 7
\end{array}
\]

Para cada una de las particiones se diseñó un algoritmo que evalúa todos los thrackles
de tamaño 9, 8, 7 y 6. Los resultados fueron los siguientes:

\begin{itemize}
  \item 9 + 8 + 8 + 8 + 3 - Probado con 9+8+8.
  \item 9 + 8 + 8 + 7 + 4 - Probado con 9+8+8.
  \item 9 + 8 + 8 + 6 + 5 - Probado con 9+8+8. 150000ms
  \item 9 + 8 + 7 + 7 + 5 - Probado con 9+8+7+6+6
  \item 9 + 8 + 7 + 6 + 6 - Probado con 9+8+7+6+6. 981709 ms.
  \item 9 + 7 + 7 + 7 + 6 - Probado con 9+7+7+7+6. 2.11354e+06 ms.
  \item 8 + 8 + 8 + 8 + 4 - Probado con 8+8+8+8. 300888 ms.
  \item 8 + 8 + 8 + 7 + 5 - Probado con 8+8+8+6+6.
  \item 8 + 8 + 8 + 6 + 6 - Probado con 8+8+8+6+6. 569735 ms.
  \item 8 + 8 + 7 + 7 + 6 - Probado con 8+8+7+7+6. 6.39485e+06 - 1 Hora, 46 minutos.
  \item 8 + 7 + 7 + 7 + 7 - Probado con 8+7+7+7+7. 1.23716e+08 - 34 Horas, 21 minutos.
\end{itemize}

En la mayoría de los casos no fue necesario examinar toda la partición, por ejemplo
para la partición $9+8+8+8+3$, encontramos que no hay 3 thrackles, para ningún
tipo de orden diferente del convexo, donde uno sea de tamaño 9
y los otros dos de tamaño 8 que sean disjuntos en aristas y por esta razón no es necesario
seguir examinando la partición a fondo.

Como para ninguna partición fue posible encontrar una descomposición en thrackles
podemos decir que el anti-thickness geométrico de $K_9$ es mayor a 5. Y como la cota
superior del anti-thickness geométrico es 6 decimos que el anti-thickness de $K_9$
es exactamente 6.
