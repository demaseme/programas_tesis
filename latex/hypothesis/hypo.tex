\documentclass[12pt, letterpaper]{article}
\usepackage[utf8]{inputenc}
\usepackage{amsmath}
\usepackage[spanish]{babel}

\newtheorem{definition}{Definición}[section]
\begin{document}
\section{Problema}
Determinar el anti-thickness geométrico de las gráficas completas de hasta 10 vértices.

\begin{definition}{Anti-thickness geométrico.}
El anti-thickness geométrico de una gráfica $G$ es la mínima $k$ tal que existe un encaje geométrico de $G$ que tiene una descomposición en $k$ thrackles.
\end{definition}
\section{Hipótesis}
\begin{enumerate}
  \item El anti-thickness geométrico para gráficas completas de hasta 10
   vértices, coincide con el valor del anti-thickness del encaje en el que
   los vértices están en posición convexa.
  \item El número de aristas de una descomposición en $k$ thrackles es a lo más
  $n - \binom{k}{2}$.
  \item El anti-thickness geométrico para gráficas completas de $n$ vértices
  en posición general,  coincide con el valor del anti-thickness
  para gráficas completas en posición convexa.
\end{enumerate}
\section{Preguntas de investigación}
\begin{enumerate}
  \item ¿Cuál  es el anti-thickness geométrico para las gráficas completas
  de hasta 10 vértices?
  \item ¿Cuál es el anti-thickness geométrico para cada tipo de orden de
  tamaño a lo sumo 10?
  \item ¿Cuál es el $\alpha$-anti-thickness geométrico para cada gráfica
  completa de hasta 10 vértices?
  \item ¿Existe siempre una descomposición en thrackles máximos para cada tipo
  de orden?
  \item ¿Cuál es el mínimo número de aristas que comparten cuales quiera 2
  thrackles máximos?
  \item ¿Cuál  es el mínimo número de aristas que comparten cuales quiera $k$
  thrackles máximos?
  \item ¿Cuál es el anti-thickness para los tipos de orden que no tienen una
  cubierta por thrackles máximos?
  \item ¿Existe un conjunto de $k$ thrackles de $m$ aristas cuya intersección
  sea vacía? Donde $m =[1,n]$.
  \item ¿Cuál es la configuración de vértices que maximiza el número
  de thrackles máximos?
\end{enumerate}
\section{Evidencias}
\subsection{Hipótesis 1}
\begin{itemize}
  \item Para analizar todos los encajes geométricos de alguna gráfica completa
  de $n$ vértices, se usan los tipos de orden para conjuntos de tamaño $n$.
  \item Se realizó una búsqueda exhaustiva para cada tipo de orden de cada $n$
  para buscar una descomposición cuyo tamaño sea menor a la dada para encajes
  en posición convexa; se encontró que el anti-thickness mínimo coincide con el
  anti-thickness de posición convexa.
\end{itemize}
\subsection{Hipótesis 2}
\begin{itemize}
  \item Es necesario determinar el mínimo número de aristas que
  comparten 2 thrackles máximos en posición general de manera geométrica y combinatoria.
\end{itemize}
\subsection{Hipótesis 3}
\begin{itemize}
  \item Si se determina que el mínimo número de aristas que comparten $k$ thrackles maximales se puede probar para los conjuntos en los que existe
  una cubierta por thrackles máximos. Para los otros conjuntos
  necesitamos probar que los conjuntos que minimizan el número de thrackles
  en la descomposición necesariamente están en posición convexa o quasi-convexa, ya que esta configuración maximiza
  el número de thrackles máximos.
\end{itemize}

\end{document}
