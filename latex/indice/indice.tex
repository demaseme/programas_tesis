\documentclass[12pt, letterpaper]{article}
\usepackage[utf8]{inputenc}
\usepackage{amsmath}
\usepackage[spanish]{babel}
\usepackage{fancyhdr}
\pagestyle{fancyplain}
\fancyhf{}
\lhead{David Merinos}
\rhead{Propuesta de índice}
\newtheorem{definition}{Definición}[section]
\begin{document}
\begin{itemize}
  \item Resumen
  \item Introducción [ Se mencionará la motivación, el problema a resolver sin ser formal, los resultados obtenidos,
  hablar de la complejidad del espacio de búsqueda, mencionar el enfoque computacional, estructura del documento]
  \item Antecedentes
    \begin{itemize}
      \item Gráfica
      \begin{itemize}
        \item Número cromático de una gráfica. [resultados del número cromático]
        \item Número de cruce de una gráfica. [resultados del número de cruce]
      \end{itemize}
      \item Gráfica geométrica
      \item Thrackles [resultados de thrackles,  Erd\H{o}s, Connway]
      \item Anti-thickness y anti-thickness geométrico. [definir el problema formalmente]
      \item Tipo de orden.
    \end{itemize}
  \item Estado del arte
    \begin{itemize}
      \item El número cromático de la gráfica de disyunción.
      \item Número cromático de gráficas geométricas.
      \item Anti-thickness geométrico del caso convexo.
      \item Anti-thickness de otros conjuntos de puntos.
    \end{itemize}
  \item Resultados
    \begin{itemize}
      \item Búsqueda de thrackles de tamaño $k$ en conjuntos en posición general.
      \item Intersección de dos thrackles máximos en posición general.
      \item Conjuntos en posición general sin thrackles máximos.
      \item Número de cruce de thrackles de algunas descomposiciones.
      \item Algoritmo exhaustivo para encontrar el anti-thickness geométrico de algún conjunto de puntos.
      \item Cota inferior para el anti-thickness geométrico en posición general.
      \item Anti-thickness geométrico de la gráfica completa de hasta 10 vértices.
    \end{itemize}
  \item Conclusiones y trabajo futuro.
  \begin{itemize}
    \item Algoritmo eficiente para obtener todas las descomposiciones.
    \item Establecer relación directa con el número de cruce y el anti-thickness geométrico.
  \end{itemize}
\end{itemize}
\end{document}
