\documentclass[12pt, letterpaper]{article}
\usepackage[utf8]{inputenc}
\usepackage{amsmath}
\usepackage[spanish]{babel}
\usepackage{fancyhdr}
\pagestyle{fancyplain}
\fancyhf{}
\lhead{David Merinos}
\newtheorem{definition}{Definición}[section]
\begin{document}
Indice:
\begin{itemize}
  \item Introducción
  \item Antecedentes
    \begin{itemize}
      \item Gráfica
      \begin{itemize}
        \item Número cromático de una gráfica.
        \item Número de cruce de una gráfica.
      \end{itemize}
      \item Gráfica geométrica
      \item Thrackles
      \item Anti-thickness y anti-thickness geométrico.
    \end{itemize}
  \item Estado del arte
    \begin{itemize}
      \item El número cromático de la gráfica de disyunción.
      \item Anti-thickness geométrico del caso convexo.
      \item Anti-thickness de otros conjuntos de puntos.
    \end{itemize}
  \item Resultados experimentales
    \begin{itemize}
      \item Búsqueda de thrackles de tamaño $k$ en conjuntos en posición general.
      \item Intersección de dos thrackles máximos en posición general.
      \item Conjuntos en posición general sin thrackles máximos.
      \item Anti-thickness geométrico de la gráfica completa de hasta 10 vértices.
      \item Número de cruce de thrackles de algunas descomposiciones.
    \end{itemize}
  \item Resultados teóricos
    \begin{itemize}
      \item Cota inferior para el anti-thickness geométrico en posición general.
      \item Anti-thickness geométrico de la gráfica completa de hasta 10 vértices.
    \end{itemize}
  \item Conclusiones y trabajo futuro.
\end{itemize}
\end{document}
