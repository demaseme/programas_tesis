\documentclass[12pt, letterpaper]{article}
\usepackage[utf8]{inputenc}
\usepackage{amsmath}
\begin{document}
\section{Verificando si \texttt{gat}$(K_9) < 6$}
Podemos verificar que \texttt{gat}$(K_9) > 4$ obteniendo todas las particiones de $\binom{9}{2} = 36$ cuyo
número de términos sea menor que $6$ y que ninguno de sus términos sea mayor a $9$.
Observamos que todas las particiones que cumplen estas condiciones son de longitud mayor a $4$.
\begin{itemize}
  \item 9 + 8 + 8 + 8 + 3 - Probado con 9+8+8.
  \item 9 + 8 + 8 + 7 + 4 - Probado con 9+8+8.
  \item 9 + 8 + 8 + 6 + 5 - Probado con 9+8+8.
  \item 9 + 8 + 7 + 7 + 5 - Probado con 9+8+7+6+6 ???
  \item 9 + 8 + 7 + 6 + 6 - Probado con 9+8+7+6+6. 981709 ms.
  \item 9 + 7 + 7 + 7 + 6 - Probando 9+7+7+7+6 en cluster.
  \item 8 + 8 + 8 + 8 + 4 - Probado con 8+8+8+8. 300888 ms.
  \item 8 + 8 + 8 + 7 + 5 - Si se prueba 8+8+8+6 se prueba este también.
  \item 8 + 8 + 8 + 6 + 6 - Falta probar 8+8+8+6 en cluster.
  \item 8 + 8 + 7 + 7 + 6 - Si se prueba 7+7+7+7 se prueba esta también.
  \item 8 + 7 + 7 + 7 + 7 - Falta probar 7+7+7+7 en cluster.
\end{itemize}

Es posible descartar alguna de estas particiones si pasa que no existen dos thrackles de tamaño $8$ que sean disjuntos
para $K_9$ en alguno de sus tipos de orden. Podemos repetir este analisis para los casos en donde haya dos o más thrackles
de tamaño $7$.
\end{document}
