\section{Conclusiones y trabajo futuro}
\begin{frame}
\frametitle{Conclusiones}
\begin{itemize}
	\item Encontramos el anti-thickness geométrico exacto para $K_n$ con $3 \leq n \leq 10$.
	\item Conjeturamos que, en general, todos los thrackles máximos comparten al menos una arista a pares y que esto se cumple para $n > 10$.
	\item Sin embargo, no nos fue posible obtener la generalización del lema que probaría el punto anterior.
	\item ¿Cómo podemos caracterizar los conjuntos de puntos con y sin thrackles máximos?
	\item Analizamos otras propiedades geométricas como el número de cruce de cada conjunto de puntos y encontramos que hay una relación con el número de thrackles máximos de cada uno.
\end{itemize}
\end{frame}

\begin{frame}
\frametitle{Trabajo futuro}
\begin{itemize}
	\item Generalizar el teorema que indica que dos thrackles máximos en posición general comparten al menos una arista para toda $n \geq 3$.
	\item Usar algoritmos genéticos para aproximar una solución del anti-thickness usando la gráfica de disyunción $D(S)$.
	\item Analizar otras propiedades geométricas como la $k$-convexidad y reflexividad de un conjunto de puntos y encontrar una relación entre estos y el anti-thickness de un dibujo.
	\item Algoritmo o eurística para encontrar el anti-thickness de un dibujo geométrico.
\end{itemize}
\end{frame}