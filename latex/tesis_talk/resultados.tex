\section{Resultados}
\begin{frame}
\frametitle{Anti-thickness geométrico}
Para $3 \leq n \leq 10$:
\[  At_g(K_n) = n - \left\lfloor\sqrt{2n + \frac{1}{4}} - \frac{1}{2} \right\rfloor \]
\end{frame}
\begin{frame}
\frametitle{Estado del arte}
Para $n \geq 3$:
\[  \frac{n-1}{2} \leq At_g(K_n) \leq n - \left\lfloor\sqrt{2n + \frac{1}{4}} - \frac{1}{2} \right\rfloor \]
Para encontrar alguna cota inferior es posible explotar alguna propiedad que se cumpla para todas las gráficas geométricas de $K_n$. Para encontrar una cota superior es posible ofrecer una descomposición de $K_n$ en thrackles.
\end{frame}
\begin{frame}
\frametitle{Estado del arte : cota inferior}
Erd\H{o}s \emph{et al.}(1988) probaron que cada gráfica geométrica con $n$ vértices en la cual no existen dos aristas disjuntas tiene a lo sumo $n$ aristas. 
\\[10pt]
Esto quiere decir que un thrackle máximo tiene a lo sumo $n$ aristas.
\end{frame}
\begin{frame}
\frametitle{Estado del arte : cota inferior}
En el trabajo de Wood \& Dujmovic se menciona que para $n \geq 3$:
\[  \frac{n-1}{2} \leq At_g(K_n).\] 
\pause

Esta cota inferior es la más sencilla, se basa en la noción del número máximo de aristas en un thrackle máximo. 
\\[5pt]
Si la gráfica completa tiene $\binom{n}{2} = \frac{n(n-1)}{2}$ aristas, ¿cuántos thrackles máximos son necesarios para \emph{cubrir} todas las aristas? Si suponemos que $k$ thrackles máximos\footnote{En el mejor caso, una descomposición por thrackles es inducida por una colección de thrackles máximos.} son necesarios la siguiente desigualdad nos otorga el resultado si resolvemos para $k$: \[ k\cdot n \geq \frac{n(n-1)}{2} \pause \Rightarrow k = \frac{n-1}{2}\]
\end{frame}

\begin{frame}
\frametitle{Estado del arte : cota superior}
Fabila-Monroy \emph{et al.} encuentran el anti-thickness exacto cuando $S$ está en posición convexa. Ellos estudian el problema del anti-thickness desde número cromático de $D(S)$. Para la cota inferior establecen el número mínimo de colores necesarios en una coloración propia de $D(S)$ y para la cota superior dan una coloración propia para cualquier $n$, con $n>3$.
\pause 
\\[10pt]
Ellos establecen que $\chi(D(S)) = n - \left\lfloor\sqrt{2n + \frac{1}{4}} - \frac{1}{2} \right\rfloor, $ cuando $S$ está en posición convexa.
\pause
\\[10pt]
Como la posición convexa es un dibujo de $K_n$ tenemos: \[At_g(K_n) \leq n - \left\lfloor\sqrt{2n + \frac{1}{4}} - \frac{1}{2} \right\rfloor. \]
\end{frame}
\begin{frame}
\frametitle{Estado del arte : thrackles máximos en posición convexa}
Un resultado del trabajo de Fabila-Monroy \emph{et al.} es que prueban que dos thrackles máximos en posición convexa siempre comparten al menos una arista. Esto significa que, en posición convexa y en el mejor caso, una colección de $k$ thrackles máximos cubre a lo sumo $kn - \binom{k}{2}$ aristas. Para obtener el valor más pequeño de $k$ podemos resolver, para $k$, la siguiente desigualdad :
\[
  kn - \binom{k}{2} \geq \binom{n}{2}.
\]
Usando la ecuación cuadrática encontramos que $k = n - \left\lfloor\sqrt{2n + \frac{1}{4}} - \frac{1}{2} \right\rfloor.$
\end{frame}
\begin{frame}
\begin{figure}
	\centering
	\includegraphics[width=0.75\linewidth]{images/thrackles_maximos}
\end{figure}
\end{frame}

\begin{frame}
\frametitle{Estado del arte : thrackles máximos en posición general}
\pause
\begin{itemize}
	\item En posición general es muy dificil dibujar thrackles máximos que sean disjuntos.
	\item La intuición nos dice que el resultado anterior es válido para posición general.
	\item ¿Cómo probamos \emph{todas} las gráficas geométricas de $K_n$?
\end{itemize}
\end{frame}

\begin{frame}
\frametitle{Tipo de orden}
Aichholzer \emph{et al.} definen el \emph{tipo de orden} de un conjunto $S=\{p_1,p_2,\dots p_n\}$ de puntos en posición general
como una función que asigna a cada tripleta ordenada $i,j,k\in\{1,2,\dots n\}$ la orientación de la tripleta de puntos $\{p_i,p_j,p_k\}$.
\begin{figure}
	\centering
	\includegraphics[width=0.3\linewidth]{images/triplet}
\end{figure}
Decimos que dos conjuntos de puntos $S_1$ y $S_2$ son combinatoriamente equivalentes cuando tienen el mismo tipo de orden.
\end{frame}

\begin{frame}
\frametitle{Tipo de orden}
Aichholzer \emph{et al.} ofrecen una base de datos para los tipos de orden de $3 \leq n \leq 10$.
\begin{table}[ht]
	\centering
	\begin{tabular}{|c|c|r|}
		\hline
		$n$ & Número de tipos de orden & Tamaño (bytes)   \\ \hline
		3     & 1                   & 6       \\ \hline
		4     & 2                   & 16      \\ \hline
		5     & 3                   & 30      \\ \hline
		6     & 16                  & 192     \\ \hline
		7     & 135                 & 1890    \\ \hline
		8     & 3315                & 53040   \\ \hline
		9     & 158817              & 5 717 412   \\\hline
		10    & 14309547            & 572 381 880 \\ \hline
	\end{tabular}
	\caption{Tipos de orden para cada $n\leq10$.}
	\label{tab:ots}
\end{table}
\end{frame}
\begin{frame}
\frametitle{Estado del arte : construyendo la nueva cota inferior}
Para analizar cada par de thrackles máximos en algun dibujo de $K_n$, primero hay que encontrarlos. Por ello, construimos un algoritmo 
exhaustivo que usa \emph{backtracking} para encontrar thrackles de cualquier tamaño. Nosotros llamamos $k-$thrackle a un thrackle de tamaño $k$.


\end{frame}
\begin{frame}
\frametitle{Algoritmo de búsqueda de $k-$thrackles}
%Explicar algoritmo de búsqueda de $k$ thrackles y cómo hacemos la intersección.
Nosotros representamos un thrackle con una tupla de aristas y cada arista está representada con un entero entre $0$ y $\binom{n}{2}-1$. Por ejemplo, si suponemos que las aristas $1,2,3$ y $8$ se intersectan a pares, entonces podemos definir un 4$-$thrackle, $T=\{1,2,3,8\}$, compuesto por las aristas antes descritas.\\[10pt]
\end{frame}
%\begin{frame}\frametitle{Algoritmo de búsqueda de $k-$thrackles}
%\begin{itemize}
%	\item Para encontrar un thrackle de 4 aristas, incializamos un \texttt{vector} de tamaño 4. \pause
%	\item Inicializamos la primera posición del vector con un $0$ y las demás con un valor nulo. \item Esto indica que el thrackle que estamos buscando actualmente, contiene a la arista $0$. $T=\{0,\thicksim,\thicksim,\thicksim\}$
%	 \item Como la arista $0$ se intersecta con el resto de las aristas en el thrackle, avanzamos a la segunda posición para aumentar el tamaño del thrackle actual. 
%	 \pause\item La segunda posición es inicializada con el siguiente número inmediato con respecto del contenido de la posición anterior. En este caso la segunda posición tendrá un valor de $1$. $T=\{0,1,\thicksim,\thicksim\}$.
%	 \item Para verificar que la arista $1$ si pertence al thrackle es necesario verificar que esta intersecte a las aristas que ya están establecidas en el thrackle.
%\end{itemize}
%\end{frame}
%\begin{frame}
%\begin{itemize}
%	\item Si la arista 1 no intersecta a la arista 0, entonces el thrackle $T={0,1,..}$ no existe, por lo tanto examinamos la siguiente arista disponible, en este caso $2$. 
%	Supongamos que este es el caso y que la arista $2$ sí intersecta a la arista $1$. $T=\{0,2,\thicksim,\thicksim\}$.
%	\item Avanzamos a la siguiente posición y repetimos el proceso anterior. $T=\{0,2,3,\thicksim\}$
%	\item Si la arista $3$ intersecta a las aristas $0$ y $2$, entonces puede quedarse en el thrackle. En otro caso debemos examinar la arista $4$. 
%	\item Repetimos este proceso iterativo hasta que lleguemos a la cuarta posición y encontremos una arista que intersecte al resto.
%\end{itemize}
%Supongamos que encontramos el thrackle $T=\{0,2,3,5\}$, lo procesamos y ahora tenemos que buscar el siguiente thrackle de tamaño 4. \pause 
%
%Para ello, incrementamos el valor de la última posición $T=\{0,2,3,6\}$ y repetimos el proceso para verificar que la arista $6$ intersecte a las aristas $0$,$2$ y $3$.
%\end{frame}
%\begin{frame}
%En caso positivo, repetimos el proceso. En caso negativo, incrementamos el valor de dicha posición. \\[10pt]\pause
%Si en algún momento, cualquiera de las posiciones alcanza el valor máximo $\binom{n}{2}-1$, entonces hacemos \emph{backtrack}. Regresamos una posición atrás, e incrementamos el valor de esa posición en uno. Esto es porque ya explotamos todas las posibles combinaciones que empiezan, por ejemplo, con $T=\{0,2,3,\thicksim\}$. En este punto el proceso continúa normalmente. $T=\{0,2,4,\thicksim\}$ \\[10pt]
%¿Condición de paro? Operando de esta manera, el algoritmo encuentra todos los thrackles de 4 aristas, que empiecen con la arista 0, luego todos los que empiecen con la arista 1, los que empiecen con la arista 2, y así sucesivamente. \pause Cuando la primera posición alcance el valor $\binom{n}{2}-1$ el algoritmo intentará hacer backtrack pero no podrá acceder a la posición $-1$, aprovechamos esta característica para detener y finalizar el algoritmo.
%\end{frame}
\begin{frame}
\centering
\includegraphics[width=0.8\linewidth]{images/example_algoritmo}
\end{frame}
\begin{frame}
\centering
\includegraphics[width=0.8\linewidth]{images/example_algoritmo2}
\end{frame}
\begin{frame}
\centering
\includegraphics[width=0.8\linewidth]{images/example_algoritmo3}
\end{frame}
\begin{frame}
\centering
\includegraphics[width=0.8\linewidth]{images/example_algoritmo4}
\end{frame}
\begin{frame}\frametitle{Costo computacional}
El peor caso es aquel en el que evaluamos todas las combinaciones de tamaño $t$ con $t \leq k$. Analizar una combinación de tamaño $t$ toma $O(t)$. Además existen $\displaystyle \binom{\binom{n}{2}}{t}$ combinaciones de tamaño $t$. Entonces necesitamos $\displaystyle t\binom{\binom{n}{2}}{t}$ operaciones para evaluar todas las combinaciones de tamaño $t$. En nuestro peor caso, $t\in[1,10]$, por lo que el costo de nuestro algoritmo en el pero caso es:

\begin{align*}
	\displaystyle O\left( \sum_{t=1}^{10} t\binom{\binom{n}{2}}{t} \right) &= 
	O\left(\binom{\binom{n}{2}}{1} + 2\binom{\binom{n}{2}}{2} + 3\binom{\binom{n}{2}}{3} +\dots+ 10\binom{\binom{n}{2}}{10} \right) = \\ &=
	 O(n^2)+ O(n^4)+ O(n^6) + \dots + O(n^{20}) = \\ &=O(n^{20})
\end{align*}

\end{frame}
\begin{frame}\frametitle{Intersección de thrackles.}
\begin{figure}
	\centering
\includegraphics[width=0.2\linewidth]{images/example_inter1}
\end{figure}
Con la representación de thrackles como tuplas de enteros, podemos hacer la operación de intersección en tiempo lineal (en el tamaño de los thrackles).
\includegraphics[width=0.2\linewidth]{images/example_inter2}~\includegraphics[width=0.2\linewidth]{images/example_inter3}~\includegraphics[width=0.2\linewidth]{images/example_inter4}~\includegraphics[width=0.2\linewidth]{images/example_inter5}~\includegraphics[width=0.2\linewidth]{images/example_inter6}

\includegraphics[width=0.2\linewidth]{images/example_inter7}
\end{frame}
\begin{frame}
\frametitle{Costo computacional - tiempos de ejecución}
En realidad, el algoritmo tomó mucho menos tiempo de lo calculado. Esto es porque el peor caso no ocurre. 
\begin{table}
	\centering
	\begin{tabular}{|c|c|c|}
		\hline
		$n$      & Tiempo teórico en cluster & Tiempo real en cluster \\ \hline
		10       &     2113.99 años          & 3 días \\   \hline
		9        &     257.01 años           & 12 minutos \\    \hline
		8        &     24.36 años            & 6 segundos \\  \hline
	\end{tabular}
\end{table}
\end{frame}
\begin{frame}
\frametitle{Estado del arte : construyendo la nueva cota inferior}
Una vez que hicimos las intersecciones de cada par de thrackles máximos en todos los dibujos de $K_n$, con $3 \leq n \leq 10$ encontramos los siguientes resultados:
\begin{itemize}
	\item Para todo tipo de orden con al menos dos thrackles máximos, cada par de thrackles máximos tienen intersección no vacía en aristas.
	\item Existen tipos de orden con solo un thrackle máximo.
	\item Existen tipos de orden en los que no hay thrackles máximos.
\end{itemize}
\end{frame}

\begin{frame}
\frametitle{Estado del arte : construyendo la nueva cota inferior}
Esto nos permite calcular el número exacto de aristas cubiertas, en el mejor caso, por una descomposición que es inducida por una colección de thrackles máximos. 

Nosotros probamos que, $m$ thrackles máximos pueden cubrir a lo sumo:
\[
-\frac{1}{2}m(m-2n-1)
\] aristas de la gráfica completa.
\pause 

\begin{table}
	\centering
\scalebox{0.7}{

	\begin{tabular}{	| >{\centering\arraybackslash}m{0.5in} | >{\centering\arraybackslash}m{0.8in} |  >{\centering\arraybackslash}m{1.2in} |  >{\centering\arraybackslash}m{1in} | }
		\hline
		$n$ & $m = \left\lceil\frac{n-1}{2}\right\rceil$ &     $-\frac{1}{2}m(m-2n-1) $ &
		$\binom{n}{2}$\\[5pt] \hline\hline
		3   & 1  & 3 & 3 \\ \hline
		4   & 2  & 7 & 6 \\ \hline
		5   & 2  & \cellcolor{red!25}9 & 10 \\ \hline
		6   & 3  & 15 & 15 \\ \hline
		7   & 3  & \cellcolor{red!25}18 & 21 \\ \hline
		8   & 4  & \cellcolor{red!25}26 & 28 \\ \hline
		9   & 4  & \cellcolor{red!25}30 & 36 \\ \hline
		10  & 5  & \cellcolor{red!25}40 & 45 \\ \hline
	\end{tabular}
}
\end{table}
\end{frame}

\begin{frame}
\frametitle{Estado del arte : construyendo la nueva cota inferior}
De la misma manera, para saber cuántos thrackles son necesarios para cubrir todas las aristas de la gráfica completa, debemos 
resolver la siguiente desigualdad para $m$:
\[
-\frac{1}{2}m(m-2n-1) \geq \binom{n}{2}.
\]
Usando la ecuación cuadrática encontramos que \[m =  n - \left\lfloor\sqrt{2n + \frac{1}{4}} - \frac{1}{2} \right\rfloor \]

\end{frame}
\begin{frame}
\frametitle{Estado del arte : construyendo la nueva cota inferior}
\begin{table}
	\centering
	\scalebox{0.9}{
		
		\begin{tabular}{	| >{\centering\arraybackslash}m{0.5in} | >{\centering\arraybackslash}m{1.6in} |  >{\centering\arraybackslash}m{1.2in} |  >{\centering\arraybackslash}m{1in} | }
			\hline
			$n$ & $m =  n - \left\lfloor\sqrt{2n + \frac{1}{4}} - \frac{1}{2} \right\rfloor $ &     $-\frac{1}{2}m(m-2n-1) $ &
			$\binom{n}{2}$\\[5pt] \hline\hline
			3   & 1  & 3 & 3 \\ \hline
			4   & 2  & 7 & 6 \\ \hline
			5   & 3  & 12 & 10 \\ \hline
			6   & 3  & 15 & 15 \\ \hline
			7   & 4  & 22 & 21 \\ \hline
			8   & 5  & 30 & 28 \\ \hline
			9   & 6  & 39 & 36 \\ \hline
			10  & 6  & 45 & 45 \\ \hline
		\end{tabular}
	}
\end{table}
Tenemos una nueva cota inferior para $3 \leq n \leq 10$:
\[ At_g(K_n) \geq n - \left\lfloor\sqrt{2n + \frac{1}{4}} - \frac{1}{2} \right\rfloor \] 
\end{frame}
\begin{frame}
\frametitle{Anti-thickness geométrico de $K_n$ para $3\leq n\leq 10$}
Con el resultado anterior y el resultado del estado del arte tenemos que, para $ 3 \leq n \leq 10$:
\[ n - \left\lfloor\sqrt{2n + \frac{1}{4}} - \frac{1}{2} \right\rfloor  \leq At_g(K_n) \leq n - \left\lfloor\sqrt{2n + \frac{1}{4}} - \frac{1}{2} \right\rfloor \] 
\pause
\[ At_g(K_n) = n - \left\lfloor\sqrt{2n + \frac{1}{4}} - \frac{1}{2} \right\rfloor \] 
\end{frame}

\begin{frame}
\frametitle{Anti-thickness geométrico de $K_n$ para $3\leq n\leq 10$}
\begin{itemize}
	\item[] Recordemos que la cota superior fue obtenida encontrando el anti-thickness de un dibujo específico, el que está en posición convexa.
	\item[] Recordemos que si $S$ es un conjunto de $n$ vértices en posición convexa, entonces \[ At_g(K(S)) = \min\{D(S)\} = \max\{D(S)\} = d(n)\]
\end{itemize}
¿Qué pasa con el anti-thickness de dibujos en posición general no convexa?
\end{frame}

\begin{frame}
\frametitle{Anti-thickness de dibujos en posición general no convexa}
Nosotros encontramos dibujos, en posición general no convexa, que tienen anti-thickness igual al anti-thickness del dibujo en posición convexa. Esto lo hicimos para cada $n$ con $3\leq n \leq 10$. Con esto podemos dar el anti-thickness geométrico exacto para $K_n$, con $n$ en el rango antes mencionado. \\[15pt]

Para encontrar estos resultados empleamos un algoritmo exhaustivo que busca descomposiciones por thrackles en la que cada elemento sea un thrackle máximo.
\end{frame}
\begin{frame}
\begin{figure}
	\centering
	\includegraphics[width=1\linewidth]{images/algo_codigo}
\end{figure}
\end{frame}

\begin{frame} \frametitle{Análisis de complejidad}
\begin{table}
	\centering
	\begin{tabular}{|c|c|c|l|}
		\hline
		$n$ & \makecell{Número máximo de \\ thrackles en un T.O.} &$At_g(K_n)$& \makecell{Número de \\      combinaciones} \\
		\hline
		6 & 5 & 3 & 10 \\ \hline
		7 & 16 & 4 & 1,820 \\\hline
		8 & 49 & 5 & 1,906,884 \\\hline
		9 & 134 & 6 & 7,177,979,809 \\\hline
		10 & 333 & 6 & 1,809,928,822,548 \\ \hline
	\end{tabular}
\end{table}
Sea $c=n - \left\lfloor\sqrt{2n + \frac{1}{4}} - \frac{1}{2}\right\rfloor$ para un tipo de orden válido, es decir, un tipo de orden con al menos $x$ thrackles máximos, de tamaño $n$, con $k$ thrackles máximos, el Algoritmo genera las
$\binom{k}{c}$ combinaciones y después, para cada una, examina, en tiempo $O(c \cdot n^2)$, si esta cubre o no a las aristas de $K_n$. Este algoritmo tiene complejidad, para un tipo de orden de $n$ puntos:
\begin{equation*}\displaystyle
O\left(\binom{k}{c}\cdot x n^2 \right) = O \left( \binom{k}{n} n^3 \right) = O(k^n n^3).
\label{complejidad_colecciones}
\end{equation*}
\end{frame}
\begin{frame} \frametitle{Resultados de la ejecución}
\begin{figure}
	\includegraphics[width=1\linewidth]{images/descomposiciones}
\end{figure}
\end{frame}
\begin{frame}
\frametitle{Anti-thickness exacto para $K_n$ con $3 \leq n \leq 10$}
Con estos dibujos cuyo anti-thickness es igual al anti-thickness de su respectivo dibujo convexo y la cota inferior descrita anteriormente, podemos dar el anti-thickness exacto para $K_n$, con $3 \leq n \leq 10$.
\begin{table}
	\centering
\begin{tabular}{|c|c|}
	\hline 
	$n$ & $At_g(K_n)$ \\ \hline
	3   &  1 \\ \hline
	4   &  2 \\ \hline
	5   &  3 \\ \hline 
	6   &  3 \\ \hline 
	7   &  4 \\ \hline
	8   &  5 \\ \hline
	9   &  6 \\ \hline
	10  &  6 \\ \hline
\end{tabular}
\end{table}
\end{frame}